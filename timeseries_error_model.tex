
%% Copernicus Publications Manuscript Preparation Template for LaTeX Submissions
%% ---------------------------------
%% This template should be used for copernicus.cls
%% The class file and some style files are bundled in the Copernicus Latex Package, which can be downloaded from the different journal webpages.
%% For further assistance please contact Copernicus Publications at: production@copernicus.org
%% https://publications.copernicus.org/for_authors/manuscript_preparation.html


%% Please use the following documentclass and journal abbreviations for preprints and final revised papers.

%% 2-column papers and preprints
\documentclass[gi, manuscript]{copernicus}



%% Journal abbreviations (please use the same for preprints and final revised papers)


% Advances in Geosciences (adgeo)
% Advances in Radio Science (ars)
% Advances in Science and Research (asr)
% Advances in Statistical Climatology, Meteorology and Oceanography (ascmo)
% Aerosol Research (ar)
% Annales Geophysicae (angeo)
% Archives Animal Breeding (aab)
% Atmospheric Chemistry and Physics (acp)
% Atmospheric Measurement Techniques (amt)
% Biogeosciences (bg)
% Climate of the Past (cp)
% DEUQUA Special Publications (deuquasp)
% Earth Surface Dynamics (esurf)
% Earth System Dynamics (esd)
% Earth System Science Data (essd)
% E&G Quaternary Science Journal (egqsj)
% EGUsphere (egusphere) | This is only for EGUsphere preprints submitted without relation to an EGU journal.
% European Journal of Mineralogy (ejm)
% Geochronology (gchron)
% Geographica Helvetica (gh)
% Geoscience Communication (gc)
% Geoscientific Instrumentation, Methods and Data Systems (gi)
% Geoscientific Model Development (gmd)
% History of Geo- and Space Sciences (hgss)
% Hydrology and Earth System Sciences (hess)
% Journal of Bone and Joint Infection (jbji)
% Journal of Environmentally Compatible Air Transport System (jecats)
% Journal of Micropalaeontology (jm)
% Journal of Sensors and Sensor Systems (jsss)
% Magnetic Resonance (mr)
% Mechanical Sciences (ms)
% Natural Hazards and Earth System Sciences (nhess)
% Nonlinear Processes in Geophysics (npg)
% Ocean Science (os)
% Polarforschung - Journal of the German Society for Polar Research (polf)
% Proceedings of the International Association of Hydrological Sciences (piahs)
% Proceedings of the International Ocean Drilling Programme (piodp)
% Safety of Nuclear Waste Disposal (sand)
% Scientific Drilling (sd)
% SOIL (soil)
% Solid Earth (se)
% State of the Planet (sp)
% The Cryosphere (tc)
% Weather and Climate Dynamics (wcd)
% Web Ecology (we)
% Wind Energy Science (wes)


%% \usepackage commands included in the copernicus.cls:
%\usepackage[german, english]{babel}
%\usepackage{tabularx}
%\usepackage{cancel}
%\usepackage{multirow}
%\usepackage{supertabular}
%\usepackage{algorithmic}
%\usepackage{algorithm}
%\usepackage{amsthm}
%\usepackage{float}
%\usepackage{subfig}
%\usepackage{rotating}


\begin{document}

%\title{Time-series uncertainty with Wiener fouling and drift}
%\title{Uncertainty of time-series measurements subject to fouling and calibration drift}
\title{Correcting time-series measurements affected by fouling and calibration drift}


% \Author[affil]{given_name}{surname}

\Author[1][thodson@usgs.gov]{Timothy O.}{Hodson} %% correspondence author
\Author[1]{Gregory E.}{Schwarz}

\affil[1]{U.S. Geological Survey Water Resources Mission Area}

%% The [] brackets identify the author with the corresponding affiliation. 1, 2, 3, etc. should be inserted.

%% If an author is deceased, please add \deceased[$Deceased date if applicable$]{$Author number$} (e.g. \deceased[13 November 2015]{2}) at the end of the affiliations. The author number depends on the placement of the author in the author list, e.g. the third author has number 3.


%% If authors contributed equally, please add \equalcontrib{$Author numbers$} (e.g. \equalcontrib{1,3}) at the end of the affiliations. The author number depends on the placement of the author in the author list, e.g. the third author has number 3.




\runningtitle{Fouling and Drift Correction}

\runningauthor{Hodson and Schwarz}





\received{}
\pubdiscuss{} %% only important for two-stage journals
\revised{}
\accepted{}
\published{}

%% These dates will be inserted by Copernicus Publications during the typesetting process.


\firstpage{1}

\maketitle



\begin{abstract}
    Environmental time-series measurements can accumulate systematic errors
    from fouling and calibration drift.
    Operators must therefore periodically check, clean, and recalibrate their measurement devices.
    By modeling error growth between checks as a stochastic process,
    we derive bias corrections and confidence intervals for the time series.
\end{abstract}


% \copyrightstatement{TEXT} %% This section is optional and can be used for copyright transfers.


\introduction  %% \introduction[modified heading if necessary]
%
Environmental monitoring often relies on in situ measurements
collected over extended deployment periods.
In practice, these measurements can exhibit systematic errors
from fouling and calibration drift,
as well as irreducible noise.
Consequently, the measurement device must be periodically checked against known references,
cleaned, and recalibrated, which resets the systematic error to zero.
Later, the information from these checks is used to correct
the raw measurements by linearly interpolating errors
between check measurements \citep{Wagner_2006}.
This procedure is mathematically equivalent to modeling the error
as a Wiener process conditioned on the check measurements.
Under this assumption, we derive expressions for the conditional mean and variance,
which are equivalent to Kalman filtering and smoothing for this system.
As the checks are also imprecise, they are treated as noisy observations
and their uncertainty is propagated into the time-series model.
Next, we extend the approach to multipoint calibrations
and monitoring networks of similar devices.
The result is a transparent and tractable framework for estimating
fouling and drift corrections with confidence intervals
that is consistent with standard operating procedures for environmental monitoring.


\section{Preliminaries}
%
Let $X$ be the true value of some variable
and let $Y$ be its measured value.
The absolute error $E$ and percent error $\%E$ are given by
%
\begin{equation}
    E = Y - X
    \qquad \text{and} \qquad
    \%E = 100 \times \frac{Y - X}{Y}
    \label{eq:absolute_error}
    \text{;}
\end{equation}
%
however, it is simpler to work with the fractional error,
%
\begin{equation}
    \mathcal{E} = \frac{Y - X}{Y} = \frac{\%E}{100}
    \label{eq:fractional_error}
    \text{,}
\end{equation}
%
and convert back to a percentage later.

We assume that the measurement error has three independent components---
fouling $E_f$, calibration drift $E_d$,
and measurement noise $E_m$---
that combine additively on the absolute scale,
%
\begin{equation}
    E = E_f + E_d + E_m
    \text{.}
\end{equation}
%
or multiplicatively on the fractional scale,
%
\begin{equation}
    1 + \mathcal{E} = (1 + \mathcal{E}_f)(1 + \mathcal{E}_d)(1 + \mathcal{E}_m)
    \text{;}
\end{equation}
%
however, for small errors (less than a few percent),
the fractional relation is approximately additive,
%
\begin{equation}
    \mathcal{E} \approx \mathcal{E}_f + \mathcal{E}_d + \mathcal{E}_m
    \text{.}
\end{equation}
%

Fouling is any physical, chemical, or biological buildup on the measurement device
that causes systematic error.
Under steady-state conditions,
the fouling error is assessed by comparing measurements
taken immediately before and after cleaning the device,
%
\begin{equation}
    E_f = Y_b - Y_a
    \qquad \text{or} \qquad
    \mathcal{E}_f = \frac{Y_b - Y_a}{Y_b}
    \label{eq:fouling_error_simple}
    \text{,}
\end{equation}
%
where $Y_b$ and $Y_a$ are the before-and-after-cleaning measurements, respectively.
Otherwise, a reference device is needed to adjust for any environmental change
during the cleaning process,
%
\begin{equation}
    E_f = (Y_b - Y_a) - (Z_b - Z_a)
    \qquad \text{or} \qquad
    \mathcal{E}_f = \frac{(Y_b - Y_a) - (Z_b - Z_a)}{Y_b}
    \label{eq:fouling_error}
    \qquad \text{\citep[][but opposite sign]{Wagner_2006},}
\end{equation}
%
where $Z_b$ and $Z_a$ are the before-and-after measurements
from a co-located reference device.
Note that the reference flips the sign of the error,
whereas we follow the convention in statistics.

Any systematic bias that remains after cleaning is attributed to calibration drift,
which is measured by comparing a clean measurement against a known standard.
Let $Z_s$ be the standard value and let $Y_s$ be the clean measurement of $Z_s$,
then the calibration drift error is given by
%
\begin{equation}
    E_d = Y_s - Z_s
    \qquad \text{or} \qquad
    \mathcal{E}_{d} = \frac{Y_s - Z_s}{Y_s}
    \label{eq:drift_error}
    \qquad \text{\citep[][but opposite sign]{Wagner_2006}.}
\end{equation}
%

Lastly, the residual error after cleaning and recalibrating the device
is attributed to measurement noise,
which is modeled as a Gaussian distribution,
%
\begin{equation}
    E_m \sim \mathcal{N}(0 ,\, \sigma_m^2)
    \qquad \text{or} \qquad
    \mathcal{E}_m \sim \mathcal{N}(0 ,\, \sigma_m^2)
    \text{,} \quad \text{where} \quad
    \sigma_m := \frac{\sigma_{\%m}}{100}
    \quad \text{for} \quad \mathcal{E}_m
    \label{eq:measurement_error}
    \text{,}
\end{equation}
%
where $\sigma_m$ denotes the measurement-device precision
and $\sigma_m^2$ is the variance of the measurement error,
%
\begin{equation}
    v_m :=
    \operatorname{Var}[E_m] = \sigma_m^2
    \label{eq:measurement_variance}
    \text{.}
\end{equation}
%
In practice, the device precision is determined under controlled conditions
and reported in its specifications
(sometimes called the ``accuracy'')
as the $2\sigma_m$ or 95\% confidence interval.

Each error component can be modeled on either scale---absolute or fractional---
but we typically use whatever scale was specified for the precision.

\section{Wiener-Process Error Model}

We model the drift and fouling errors as a Wiener process
with drift.
A Wiener process is a continuous-time stochastic process
with independent, normally distributed increments
whose variance grows linearly with elapsed time.
Specifically, for a process $\{X_t\}_{t \ge 0}$
with drift $\mu$ and diffusion coefficient $\sigma$,
the increment over time $t$ satisfies
%
\begin{equation}
    X_t - X_0 \sim
    \mathcal{N}\!\left(\mu t,\; \sigma^2 t\right)
    \qquad \text{for } t \ge 0
    \qquad
    \text{\citep{Taylor_1998}.}
\end{equation}
%
The drift term $\mu$ represents the mean rate of change
(with $\mu=0$ for a classical Wiener process)
and is important for devices that exhibit systematic bias,
such as a device that gradually loses responsiveness
due to the degradation of a consumable component,
like a reagent in a colorimetric analyzer.

Let $E(t)$ denote the measurement error at time $t$,
modeled as a continuous-time stochastic process,
and let $e(t)$ denote a realization of this process.
Observations are available only at a finite set of times
when fouling or calibration checks are performed,
$0 = t_0 < t_1 < \cdots < t_K = T$.
For notational convenience, we define
$E_k := E(t_k)$
and
$e_k := e(t_k)$.
Thus, $t_k - t_{k-1}$ denotes the elapsed time between successive error
observations $e_{k-1}$ and $e_k$.
Throughout, the index $k$ denotes check times $t_k$, and subscripts $f$ and $d$
denote the fouling and calibration drift components, respectively.

We first consider the period after observing the error at time $t_{k-1}$
but before observing the error at time $t_k$.
We refer to this as an ``open interval,'' which arises during real-time
uncertainty estimation.
In contrast, a ``closed interval'' is bracketed by observations at both
$t_{k-1}$ and $t_k$; in this case, the conditional mean and variance are
updated after incorporating the observation at $t_k$.

For an open interval,
the conditional distribution of the Wiener-process error is
%
\begin{equation}
    p(e_t \mid E_{k-1} = e_{k-1})
    =
    \mathcal{N}(m^o(t),\, v^o(t))
    \text{,} \qquad
    t > t_{k-1}
    %\mathcal{N}(\delta_{k-1}  e_{k-1} + \mu t ,\, \sigma^2 t)
\end{equation}
%
with conditional mean and variance given by
%
\begin{align}
    m^o(t) &:=
        \mathbb{E}\left[E_t \mid E_{k-1} = e_{k-1} \right]
        = \delta_{k-1} e_{k-1} + \mu (t - t_{k-1})
        \qquad \text{and}
        \label{eq:wiener_mean}\\
    v^o(t) &:=
        \operatorname{Var}\left[E_t \mid E_{k-1} = e_{k-1} \right]
        = \sigma^2 (t - t_{k-1})
        \label{eq:wiener_variance}
    \text{.}
\end{align}
%
where $\delta_{k-1}$ indicates whether the error process has been reset
due to cleaning or recalibration,
%
\begin{equation}
    \delta_{k-1} =
    \begin{cases}
        0, & \text{reset at } t_{k-1} \\
        1, & \text{otherwise}
    \end{cases}
    \text{,}
\end{equation}
%
$e_{k-1}$ is the error at time $t_{k-1}$,
%$\mathcal{N}$ is the normal distribution
%specified by its mean and variance,
$\mu$ is the drift rate (mean change per unit time),
$\sigma^2$ is the instantaneous variance rate (variance per unit time).

Later on, upon observing the error at time $t_k$,
the conditional distribution of $E_t$ is updated.
A Wiener process conditioned to start and end at specified values
is known as a Brownian bridge \citep{Taylor_1998}.
To simplify notation, let $\lambda_k(t)$ be the linear interpolation weight,
%
\begin{equation}
    \lambda_k(t) = \frac{t - t_{k-1}}{t_k - t_{k-1}}
    \text{,} \qquad
    t \in [t_{k-1}, t_k]
    \text{.}
\end{equation}
%
The conditional distribution of the bridge at time
$t$ is
%
\begin{equation}
    p(e_t \mid E_{t_{k-1}} = e_{k-1},\, E_{t_k} = e_k)
    =
    \mathcal{N} \left(
        m^c(t),\, v^c(t)
        %\delta_{k-1} \left( 1 - \lambda(t) \right)e_{k-1}
        %+ 
        %\lambda(t) e_k,\;
        %\sigma^2 (1 - \lambda(t)) \lambda(t) (t_k - t_{k-1})
    \right)
    \text{,} \qquad
    t \in \left[t_{k-1},\, t_k \right]
\end{equation}
%
with conditional mean and variance given by
%
\begin{align}
    m^c(t) &:=
        \mathbb{E}\left[ E_t \mid E_{t_{k-1}} = e_{k-1} ,\, E_{t_k} = e_k \right]
        =
        \delta_{k-1} \left( 1 - \lambda(t) \right) e_{k-1} + \lambda(t) e_k
        \qquad \text{and}
        \label{eq:bridge_mean} \\
    v^c(t) &:=
    \operatorname{Var} \left[ E_t \mid E_{t_{k-1}} = e_{k-1} ,\, E_{t_k} = e_k \right]
        = \sigma^2 \frac{(t - t_{k-1})(t_{k} - t)}{t_{k} - t_{k-1}} 
        % = \sigma^2 (1 - \lambda_k(t)) (t - t_{k-1})
        \label{eq:bridge_variance}
    \text{.}
\end{align}
%
Let $\lambda^+_k(t)$ be the linear interpolation weight,
extended by
%
\begin{equation}
    \lambda^+_k(t) :=
    \begin{cases}
        1, & t < t_{1} \\
        \frac{t - t_{k-1}}{t_k - t_{k-1}}, & t \in [t_{k-1}, t_k] \\
        0, & t > t_{K}
    \end{cases}
    \text{,}
\end{equation}
%
such that $\lambda^+_k(t) = 1$ or $0$ when extrapolating
below or above the standard range, respectively.
Also let $\phi^+_k(t)$ define the bridge factor, extended by
%
\begin{equation}
    \phi^+_k(t) :=
    \begin{cases}
        (t_{1} - t), & t < t_{1} \\
        \frac{(t - t_{k-1})(t_{k} - t)}{(t_{k} - t_{k-1})},
        & t \in [t_{k-1}, t_{k}] \\
        (t - t_{K}), & t > t_{K} 
    \end{cases}
    \text{.}
\end{equation}
%
By substituting these definitions into the equations for the mean and covariance,
the open and closed forms become equivalent,
%
\begin{align}
    m_k(t) &= \delta_{k-1} \left( 1 - \lambda^+_k(t) \right)e_{k-1} + \lambda^+_k(t) e_k
    \label{eq:unified_wiener_mean}
    \qquad \text{and} \qquad \\
    v_k(t) &= \sigma^2 \phi^+_k(t)
    \label{eq:unified_wiener_variance}
    \text{,}
\end{align}
%
thereby eliminating the need for superscripts ($m^o$, $m^c$, etc.).

\section{Check-Measurement Uncertainty}

Check measurements contribute additional uncertainty to the error process.
We review the general approach for propagating check measurement uncertainty,
then derive the check uncertainty for fouling and calibration drift.

For a bridge with uncertain endpoints,
define the endpoint expectations and variances as
%
\begin{equation}
    m_{k}(t_{k-1}) := \mathbb{E}\left[ E_{k-1}\right]
    \text{,} \qquad
    m_{k}(t_k) := \mathbb{E}\left[ E_{k}\right]
    \text{,} \qquad
    v_{k}(t_{k-1}) := \operatorname{Var}\left[ E_{k-1}\right]
    \text{,} \qquad
    v_k(t_k) := \operatorname{Var}\left[ E_{k}\right]
\end{equation}
%
Given the Brownian-bridge representation,
%
\begin{equation}
    E_t = (1 - \lambda^+_k(t)) E_{k-1} + \lambda^+_k(t) E_k + W_t
    \text{,} \qquad
    W_t \sim \mathcal{N}\left(0, \sigma^2 \phi^+_k(t)\right)
    \text{.}
\end{equation}
%
by linearity of expectation,
%
\begin{align}
    \mathbb{E}[E_t] &= (1 - \lambda^+_k(t)) \, \mathbb{E}[E_{k-1}]
                        + \lambda^+_k(t) \, \mathbb{E}[E_k] + \mathbb{E}[W_t]
                        \notag \\
                    &= (1 - \lambda^+_k(t)) m_{k}(t_{k-1}) + \lambda^+_k(t) m_{k}(t_k)  
    \text{,}
\end{align}
%
and, because $W_t$ is independent of the endpoint errors $E_{k-1}$ and $E_k$,
the variance decomposes into separate terms.
Applying the variance formula for a linear combination gives
%
\begin{align}
    \operatorname{Var}[E_t]
        &= \operatorname{Var}[
            (1 - \lambda^+_k(t)) \, E_{k-1}
            +
            \lambda^+_k(t) \, E_{k}
            ]
            + \operatorname{Var}[W_t]
            &\text{(Appendix~\ref{sec:variance_linear_combination})}
            \notag \\
        &=  \underbrace{
            (1 - \lambda^+_k(t))^2 \, v_{k}(t_{k-1})
            + \lambda^+_k(t)^2 \, v_k(t_k)
            }_{\tilde{v}_k(t)}
            + 
            \underbrace{\sigma^2 \phi^+_k(t)}_{v_k(t)}
    \label{eq:variance_uncertain_bridge}
    \text{,}
\end{align}
%
where $\tilde{v}_k(t)$ represents the contribution from check-measurement uncertainty,
\begin{equation}
    \tilde{v}_k(t) := 
        (1 - \lambda^+_k(t))^2 \, v_{k}(t_{k-1})
        + \lambda^+_k(t)^2 \, v_{k}(t_k)
    \label{eq:check_variance}
    \text{.}
\end{equation}
%
If the endpoint errors have the same variance,
$v_{k}(t_{k-1}) = v_{k}(t_k)$,
then the conditional variance at time $t$ is
%
\begin{equation}
    \tilde{v}_k(t) = \left(
                        (1 - \lambda^+_k(t))^2 + \lambda^+_k(t)^2
                      \right) \, v_{k}(t_k)
    \label{eq:check_constant_variance}
    \text{,}
\end{equation}
%
For an open interval, this simplifies to
$ \tilde{v}^o_k(t) = v_{k}(t_k)$.

The following sections derive expressions for the
fouling and calibration drift check uncertainties,
$v_{d}(t_d)$ and $v_{f}(t_f)$,
to use in Equation~\ref{eq:check_constant_variance}.
Later, we introduce multipoint calibrations,
for which the endpoint variances may differ;
in that case, we use Equation~\ref{eq:check_variance} for the check-measurement variance.

\subsection{Fouling-Check Uncertainty}
% TODO: Introduce v_f(t_f) into this section and the next.

For the fouling check, the measurement $m$ and reference $r$
devices have known precisions $\sigma_m$ and $\sigma_r$, 
with normally distributed, independent errors,
\begin{equation}
    E^{m} \sim \mathcal{N}\left(0,\, \sigma^2_m \right)
    \text{,} \qquad
    E^{r} \sim \mathcal{N}\left(0,\, \sigma^2_r \right)
    \text{,} \qquad
    E^{m} \perp E^{r}
    \text{.}
\end{equation}
%
Let $Y$ and $Z$ be the principal and reference measurements,
and let $X^m$ and $X^r$ be their respective true values,
%
\begin{equation}
    Y = X^m + E^{m}
    \text{,} \qquad
    Z = X^r + E^{r}
    \text{.}
\end{equation}
%
The superscripts are necessary because the values of $X$
depend on the calibration of either device.
Substituting back into the fouling error (Equation~\ref{eq:fouling_error})
gives
%
\begin{align}
    E_f &= (Y_b - Y_a) - (Z_b - Z_a) \notag \\
        &= \left((X^m_b + E^m_{b}) - (X^m_a + E^m_{a})\right)
           - \left((X^r_b + E^r_{b}) - (X^r_a + E^r_{a})\right) \notag \\
        &= (E^m_{b} - E^m_{a}) - (E^r_{b} - E^r_{a})
    \text{.}
\end{align}
%
By independence ($E^{m} \perp E^{r}$), the variance is
\begin{align}
    \operatorname{Var}[E_f]
        &= \operatorname{Var}[E^m_b - E^m_a]
          + \operatorname{Var}[E^r_b - E^r_a]
    \text{.}
\end{align}
%
If the measuring device uses internal filtering
(e.g., time averaging),
such that the errors are perfectly correlated during the check,
then the before-and-after errors cancel and the measurement variance is zero,
%
\begin{equation}
    \operatorname{Corr} \left( E^{m}_{a},\, E^{m}_{b} \right) = 1
    \quad \Rightarrow \quad
    \operatorname{Var}[E^m_b - E^m_a] = 0
    \text{,}
\end{equation}
%
Similarly, if the reference device also uses filtering;
otherwise, if its errors are independent,
then the measurement variance is given by
%
\begin{equation}
    \operatorname{Corr}(E_{z_a},\, E_{z_b}) = 0
    \quad \Rightarrow \quad
    \operatorname{Var}[E_f]
    = \operatorname{Var}[E^r_b] + \operatorname{Var}[E^r_a]
    = \sigma^2_r + \sigma^2_r = 2 \sigma^2_r
    \text{.}
\end{equation}
%
Therefore, the fouling-check variance is either zero or $2 \sigma^2_r$,
depending on the correlation structure of the reference device.

\subsection{Calibration-Check Uncertainty}

For a calibration check, 
the standard and measurement have known precisions $\sigma_s$ and $\sigma_m$
and independent errors,
%
\begin{equation}
    E^{s} \sim \mathcal{N}\left(0,\, \sigma^2_s \right)
    \text{,} \qquad
    E^{m} \sim \mathcal{N}\left(0,\, \sigma^2_m \right)
    \text{,} \qquad
    E^{s} \perp E^{m}
    \text{.}
\end{equation}
%
Let $Y_s$ and $Z_s$ be the measurement and the standard,
and let $X_s$ be the true value of the standard,
%
\begin{equation}
    Y_s = X_s + E^{m}
    \text{,} \qquad
    Z_s = X_s + E^{s}
    \text{.}
\end{equation}
%
Substituting back into the drift error
(Equation~\ref{eq:drift_error}) gives
%
\begin{equation}
    E_d = (X_s + E^m) - (X_s + E^s)
    \text{,}
\end{equation}
%
and by independence ($E^{m} \perp E^{s}$), the variance of the drift check is
%
\begin{equation}
    \operatorname{Var}[E_d]
        = \operatorname{Var}[E^m] + \operatorname{Var}[E^s]
        = \sigma^2_m + \sigma^2_s
    \text{.}
\end{equation}
%

\section{Multipoint Calibration}

In a multipoint calibration, a device is checked against multiple standards
that should span the expected range of the measurand.
This process is essential when the measurement response is nonlinear
and the magnitude of drift varies across the range.
For example, a pH meter might be calibrated using standards at pH 4, 7, and 10.
For any specific value, the drift correction is determined by linearly
interpolating between the corrections of the two adjacent standards
\citep{Wagner_2006}.

Let $z_s$ denote the standard values for $s = 1, 2, \ldots, S$.
For a given check time $t_k$,
the error is determined at each standard value,
%
\begin{equation}
    E_{k,s} := E(t_k,\, z_s) = e_{k,s}
    \text{,} \qquad
    \forall s \in \{1, 2, \ldots, S\}
    \text{.}
\end{equation}
%
First, we consider the conditional distribution at the check time $t_{k}$,
which forms a bridge in the $y$ dimension.
%In the multipoint calibration,
Next, we will use that first bridge as the uncertain boundary condition
for the bridge in the time dimension.

Given check-measurement errors $e_{k,s-1}$ and $e_{k,s}$,
the conditional mean of the $y$-bridge at time $t_k$ is given by
%
\begin{align}
m_{k}(y) &:= \mathbb{E} \left[
                            E_{k,y}
                            \mid
                            E_{k,s-1} = e_{k,s-1},\,
                            E_{k,s} = e_{k,s}
                           \right]
            \notag \\
            &= (1 - \lambda^+_s(y)) \, \mathbb{E} [E_{k,s-1}]
             + \lambda^+_s(y) \, \mathbb{E} [E_{k,s}] \notag \\
            &= (1 - \lambda^+_s(y)) e_{k,s-1}
             + \lambda^+_s(y) e_{k,s}
\label{eq:y_bridge_mean}
\text{,}
\end{align}
%
which is a linear interpolation between the check-measurement errors.
%
The conditional variance of the $y$-bridge at time $t_k$ is
%
\begin{align}
    v_{k}(y) &:= \operatorname{Var} \left[
                            E_{k,y}
                            \mid
                            E_{k,s-1} = e_{k,s-1},\,
                            E_{k,s} = e_{k,s}
                            \right] \notag \\
               &= (1 - \lambda^+_s(y))^2 \, \operatorname{Var}[E_{k,s-1}]
                 + \lambda^+_s(y)^2 \, \operatorname{Var}[E_{k,s}]
                + \eta^2 \phi^+_{s}(y)
    \label{eq:y_bridge_variance}
    \text{,}
\end{align}
%
where $\eta$ is the diffusion coefficient in the $y$ dimension.

Next, let these mean and variance define the boundary conditions
(end points) for the time-bridge.
The conditional mean is given by
the bilinear interpolation between four check points,
$e_{k-1,s-1}$, $e_{k-1,s}$, $e_{k,s-1}$, and $e_{k,s}$,
denoted as the set
$\{e_{i,j}\}$ where $i\in\{k-1,k\}$ and $j\in\{s-1,s\}$.
%
\begin{align}
    m_k(t, y)
    &:=
    \mathbb{E}\left[
        E_{t,y}\,\mid\,
        \{\,E_{t_i,y_j}=e_{i,j}\,\}_{\,i\in\{k-1,k\},\, j\in\{s-1,s\}}
        \right] \notag \\
    &=
    (1-\lambda^+_k(t),\;\lambda^+_k(t))
    \begin{bmatrix}
        \delta_{k-1} e_{k-1,s-1} & \delta_{k-1} e_{k-1,s} \\
        e_{k,s-1}   & e_{k,s}
    \end{bmatrix}
    \begin{bmatrix}
        1-\lambda^+_s(y) \\
        \lambda^+_s(y)
    \end{bmatrix}
    \label{eq:multi_check_mean}
    \text{.}
\end{align}
%
Expanding the right-hand side yields
%
\begin{align}
    m_k(t,y)
    &=
    (1 - \lambda_k(t))(1 - \lambda_s(y)) \delta_{k-1} e_{k-1,s-1} \notag \\
    &\quad
    + (1 - \lambda_k(t)) \lambda_s(y) \delta_{k-1} e_{k-1,s} \notag \\
    &\quad
    + \lambda_k(t)(1 - \lambda_s(y)) e_{k,s-1} \notag \\
    &\quad
    + \lambda_k(t) \lambda_s(y) e_{k,s}
    \text{,}
\end{align}
%
which satisfies the conditions for the four check points,
%
\begin{align}
    m_k(t_{k-1}, y_{s-1}) &=  \delta_{k-1}e_{k-1,s-1} \text{,} \notag \\
    m_k(t_{k-1}, y_{s})   &=  \delta_{k-1}e_{k-1,s} \text{,} \notag \\
    m_k(t_{k}, y_{s-1})   &=  e_{k,s-1} \text{,} \notag \\
    m_k(t_{k}, y_{s})     &=  e_{k,s} \text{.}
\end{align}
%
To derive the conditional variance,
substitute the variance of the $y$-bridge
(Equation~\ref{eq:y_bridge_variance})
into the equation for a time bridge with uncertain endpoints
(Equation~\ref{eq:variance_uncertain_bridge})
%
\begin{align}
    \operatorname{Var}\left[
        E_{t,y}\,\mid\,
        \{\,E_{t_i,y_j}=e_{i,j}\,\}_{\,i\in\{k-1,k\},\, j\in\{s-1,s\}}
        \right]
    =  
    \underbrace{
        (1 - \lambda^+_k(t))^2 v_{k-1}(y)
        + \lambda^+_k(t)^2 v_k(y)
    }_{\tilde{v}_k(t,y)}
    + 
    \underbrace{\sigma^2 \phi^+_k(t)}_{v_k(t)}
\end{align}
%
where $\tilde{v}_k(t,y)$ represents the contribution from check-measurement uncertainty,
%
\begin{equation}
    \tilde{v}_k(t,y) :=
        \left((1 - \lambda^+_k(t))^2,\; \lambda^+_k(t)^2\right)
        \begin{bmatrix}
            \operatorname{Var}[E_{k-1,s-1}] & \operatorname{Var}[E_{k-1,s}] \\
            \operatorname{Var}[E_{k,s-1}]   & \operatorname{Var}[E_{k,s}]
        \end{bmatrix}
        \begin{bmatrix}
            (1 - \lambda^+_s(y))^2 \\
            \lambda^+_s(y)^2
        \end{bmatrix}
        + \left((1 - \lambda^+_k(t))^2 + \lambda^+_k(t)^2\right)\eta^2 \phi^+_s(y)
    \label{eq:multi_check_variance}
    \text{.}
\end{equation}
%
% Note that the standards need not be consistent between checks.

\section{Combined Error}

For each error component, $k\in\{f,d\}$,
we decompose the conditional variance into a process term
(bridge uncertainty) and a check-measurement term:
%
\begin{equation}
\operatorname{Var}\!\left[E_k\right] = v_k(t) + \tilde v_k(t),
\end{equation}
%
where the process-only variance is
$
v_k(t) := \sigma_k^2\,\phi_k^+(t),
$
and $\tilde v_k(t)$ is the variance from the check measurements propagated to time $t$
(Equations~\ref{eq:check_variance} or \ref{eq:multi_check_variance}).

For absolute errors,
the combined conditional mean and variance are
%
\begin{align}
m(t) &=
    m_{f}(t)
    +
    m_{d}(t, y)
    \label{eq:total_mean}
    \qquad \text{and} \\
v(t) &=
    v_{f}(t)
    +
    \tilde{v}_{f}(t)
    +
    v_{d}(t)
    +
    \tilde{v}_{d}(t, y)
    +
    v_{p}
    \label{eq:total_variance}
    \text{,}
\end{align}
%
where $f$, $d$, and $p$ denote the fouling, drift, and precision components, respectively;
the means are given by Equations~
\ref{eq:unified_wiener_mean}
and
\ref{eq:multi_check_mean};
and the variances by Equations
\ref{eq:measurement_error},
\ref{eq:unified_wiener_variance},
\ref{eq:check_variance},
and
\ref{eq:multi_check_variance}.

For fractional errors,
the combined mean and variance are
%
\begin{align}
    m(t) &= (1+m_f)(1+m_d) - 1
    \qquad \text{and} \\
    v(t) &=
        \left(v_f + (1+m_f)^2\right)
        \left(v_d + (1+m_d)^2\right)
        \left(v_p + 1\right)
        - \left((1+m_f)(1+m_d)\right)^2
    \qquad \text{(Appendix~\ref{sec:combined_fractional_proof}).}
\end{align}
%

% TODO DESCRIBE FIGURE HERE

%\begin{figure*}[t]
%\includegraphics[width=12cm]{FILE NAME}
%\caption{TEXT}
%\label{fig:}
%\end{figure*}
%

\section{Corrected Measurement and Confidence Interval}

At a fixed time $t$, recall the observation model $Y = X + E$.
Let $m(t)$ and $v(t)$ denote the mean and variance of $E$ at time $t$.
Treating $X$ as fixed and $E$ as independent measurement error,
the bias-corrected estimate of $X$ is
%
\begin{equation}
    x(t) := y - m(t),
\end{equation}
%
with an approximate $(1-\alpha)$ confidence interval
%
\begin{equation}
    \mathrm{CI} = x(t) \pm q_{1-\alpha/2}\sqrt{v(t)}.
\end{equation}
%

For fractional errors $\mathcal{E}$,
we use the observation model $Y = X(1+\mathcal{E})$.
Assuming $\mathcal{E}$ is independent of $Y$
with mean $m(t)$ and variance $v(t)$,
a bias-corrected estimate of $X$ is
%
\begin{equation}
x(t) := y\bigl(1-m(t)\bigr)
\text{,}
\end{equation}
%
with approximate confidence interval
%
\begin{equation}
\mathrm{CI} = x(t) \pm q_{1-\alpha/2}\,y\sqrt{v(t)}
\text{.}
\end{equation}
%

When measurement errors exceed 100\% and the measurand is strictly positive,
we also provide the log-multiplicative error model
in Appendix~\ref{sec:log_multiplicative_error}.

\section{Parameter Estimation}

Let $r_k$ be the one-step-ahead residual and $u_k$ its conditional variance
%
\begin{align}
    r_k &= e_k - m^o_k(t_k) \\
    u_k &= v^o_k(t_k) + v_k + v_r
    \text{,}
\end{align}
%
where $v_k$ denotes the observation error variance at time $t_k$,
and $v_r$ denotes the observation error variance at the most recent reset event
(cleaning or recalibration).
The drift rate $\mu_k$ and diffusion coefficient $\sigma_k$
can be estimated by maximizing the log-likelihood function
%
\begin{equation}
    \ell(\mu_k, \sigma_k^2)
    =
    - \frac{1}{2} \sum_{k=1}^{K_i}
    \left[
        \log \left( u_k \right)
        +
        \frac{r_k^2}{u_k}
    \right]
    \text{.}
\end{equation}
%

For the special case of no observational uncertainty,
$v_k = v_r \equiv 0$,
the drift and diffusion parameters have closed-form maximum-likelihood estimators.
Specifically,
%
\begin{equation}
    \hat{\mu}_k = \frac{\sum_{k=1}^{K} e_k - \delta_{k-1} e_{k-1}}
                       {\sum_{k=1}^{K} (t_k - t_{k-1})}
\end{equation}
%
and
%
\begin{equation}
    \hat{\sigma}_k^2 = \frac{1}{K}{\sum_{k=1}^{K} \frac{r_k^2}{t_k - t_{k-1}}}
    \text{.}
\end{equation}
%
Beware that when observational uncertainty is ignored,
these estimators become ill-conditioned for short intervals,
because the denominator goes to zero
but the numerator does not when $t_k - t_{k-1} \rightarrow 0$.
% When a closed-form expression is preferred,
% the observational uncertainty can be included using the following
% truncated heuristic,
% %
% \begin{equation}
%     \hat{\sigma}^2  =  \frac{1}{K}{\sum_{k=1}^{K}
%                        \max
%                        \left(
%                         0, \,
%                         \frac{r_k^2 - v_k - v_r}
%                              {t_k - t_{k-1}}
%                        \right)
%                      }
%     \text{.}
% \end{equation}

For the multipoint calibration,
define residual and variance terms
%
\begin{align}
    r_{d,s} &:= e_{d,s} - e_{d,s-1} \\
    u_{d,s} &:= \eta^2 (z_{d,s} - z_{d,s-1})
                + v_{d,s} + v_{d,s-1}
    \text{,}
\end{align}
%
and estimate the diffusion rate parameter $\eta$
by maximizing the log-likelihood,
%
\begin{equation}
    \ell(\eta^2)
    =
    %-\frac{K_i}{2} \ln(2 \pi)
    - \frac{1}{2} \sum_{d=1}^{D} \sum_{s=2}^{S_d}
    \left[
        \log \left( u_{d,s} \right)
        +
        \frac{r_{d,s}^2}{u_{d,s}}
    \right]
    \text{,}
\end{equation}
%
where $S_d$ is the number of standards in the $d$-th calibration check.
This is necessary because in practice,
only some of the standards may be checked at a time.

Assuming no observational uncertainty,
$v_s = v_{s-1} \equiv 0$,
the closed-form of the estimator is
%
\begin{equation}
    \hat{\eta}^2 =
    \frac{1}{N}
    \sum_{d=1}^{D} \sum_{s=2}^{S_d}
    \frac{\left(e_{d,s} - e_{d,s-1}\right)^2}
         {z_s - z_{s-1}}
    \text{,}
\end{equation}
%
where $N = \sum_{d=1}^{D} (S_d - 1)$.
This estimator is better conditioned because
the calibration standards are separated in the $y$-dimension,
ensuring that the denominator never approaches zero.

\section{Monitoring Networks}

% \sigma_ij (site i and bin j )
%
Inverse gamma distribution prior
%
\begin{equation}
    p(\sigma^2 \mid a ,\; b)
    =
    \frac{b^a}{\Gamma(a)} (\sigma^2)^{-(a+1)}
    \exp\left(-\frac{b}{\sigma^2}\right)
\end{equation}
%
\begin{equation}
    \bar{\sigma}
    \equiv
    \mathbb{E} \left[ \sigma^2 \mid a, b \right]
    = \frac{b}{a - 1}
\end{equation}
%
% TODO define residuals as r and sites as j
\begin{equation}
    \ln L(r \mid a ,\; b)
    =
    \sum_{j \in J}
    \left(
        - \frac{N_j}{2} \ln(2 \pi)
        + a \ln(b)
        + \ln \Gamma \left( \frac{N}{2} + a \right)
        - \left( \frac{N_j}{2} + a \right)
        \ln( \frac{N_j s^2_j}{2} + b)
        - \ln \Gamma(a)
    \right)
\label{eq:hierarchical_likelihood}
\end{equation}
%

The hierarchical estimate of the instantaneous variance rate $\sigma^2$ for site $j$ is given by
%
\begin{equation}
    \hat{\sigma}^2_j
    =
    \hat{w}_j s_j^2 + (1 - \hat{w}_j) \bar{\hat{\sigma}}^2
\end{equation}
%
where
%
\begin{equation}
    \hat{w}_j
    =
    \frac{N_j}{N_j + 2 (\hat{a} - 1)}
\end{equation}
%

\section{Data Requirements}

Implementing the bias-correction framework requires the following information.
%
\begin{enumerate}
    \item The time and value of each check measurement,
    including any associated reference measurements and standard values.
    \item The time of each cleaning and recalibration (when the error process resets).
    \item The precision for the measurement device and any reference devices or standards.
    \item The device type (make and model) for grouping like devices in hierarchical estimation.
    \item A unique device identifier for tracking devices across redeployments.
    \item The error scale (absolute or fractional) for each error component.
\end{enumerate}
%

\conclusions  %% \conclusions[modified heading if necessary]

We present an error model for time-series measurements
affected by fouling and calibration drift.
Our model is consistent with standard bias-correction procedures,
but by formalizing the assumptions into a stochastic process,
we derive confidence intervals for the corrected time series.
We extend this framework to multipoint calibrations
and to pooling information across monitoring networks of similar devices.

%% The following commands are for the statements about the availability of data sets and/or software code corresponding to the manuscript.
%% It is strongly recommended to make use of these sections in case data sets and/or software code have been part of your research the article is based on.

% \codeavailability{TEXT} %% use this section when having only software code available
% \dataavailability{TEXT} %% use this section when having only data sets available
% \codedataavailability{TEXT} %% use this section when having data sets and software code available
% \sampleavailability{TEXT} %% use this section when having geoscientific samples available
% \videosupplement{TEXT} %% use this section when having video supplements available


\appendix

\section{Variance of a Linear Combination}
\label{sec:variance_linear_combination}

For any random variables $X_1, X_2, \ldots, X_n$ with finite second moments,
and any constants $a_1, a_2, \ldots, a_n$,
%
\begin{equation}
    Y = a_1 X_1 + a_2 X_2 + \ldots + a_n X_n
\end{equation}
%
the variance is given by
%
\begin{equation}
    \operatorname{Var}(Y)
    =
    \sum_{i=1}^{n} \sum_{j=1}^{n} a_i a_j \operatorname{Cov}(X_i, X_j)
\end{equation}
%
In the two-variable case,
%
\begin{equation}
    Y = a X + b Z
\end{equation}
%
then the variance is
%
\begin{equation}
    \operatorname{Var}(Y) = a^2 \operatorname{Var}(X) + b^2 \operatorname{Var}(Z) + 2ab \operatorname{Cov}(X, Z)
\end{equation}



\section{Combined mean and variance for independent fractional errors}
\label{sec:combined_fractional_proof}

Let $E_f$, $E_d$, and $E_p$ denote independent fractional-error components with
means $m_f,m_d,0$ and variances $v_f,v_d,v_p$, respectively, and define the
combined fractional error $E$ by
%
\begin{equation}
    1+E := (1+E_f)(1+E_d)(1+E_p)
    \text{.}
\end{equation}
%
Then
%
\begin{equation}
    m := \mathbb{E}[E] = (1+m_f)(1+m_d)-1
    \text{,}
\end{equation}    
%
and
%
\begin{equation}
    v := \operatorname{Var}(E)
        =
        \left(v_f+(1+m_f)^2\right)
        \left(v_d+(1+m_d)^2\right)
        \left(v_p+1\right)
        -\left((1+m_f)(1+m_d)\right)^2
        \text{.}
\end{equation}
%

\paragraph{Proof.}
%
Let $A=1+E_f$, $B=1+E_d$, and $C=1+E_p$, so that $1+E=ABC$.
By independence,
%
\begin{equation}
    \mathbb{E}[ABC]=\mathbb{E}[A]\mathbb{E}[B]\mathbb{E}[C]
                   = (1+m_f)(1+m_d)\cdot 1,
\end{equation}
%
which implies $m=\mathbb{E}[E]=\mathbb{E}[ABC]-1=(1+m_f)(1+m_d)-1$.
%
Also by independence,
%
\begin{equation}
    \operatorname{Var}(ABC)=\mathbb{E}[A^2B^2C^2]-\mathbb{E}[ABC]^2
    =\mathbb{E}[A^2]\mathbb{E}[B^2]\mathbb{E}[C^2]-\mathbb{E}[ABC]^2
    \text{.}
\end{equation}
%
Using $\mathbb{E}[A^2]=\operatorname{Var}(A)+\mathbb{E}[A]^2=v_f+(1+m_f)^2$,
$\mathbb{E}[B^2]=v_d+(1+m_d)^2$, and $\mathbb{E}[C^2]=v_p+1$ (since
$\mathbb{E}[C]=1$), we obtain the stated expression for $v$.


\section{Log-multiplicative error model}
\label{sec:log_multiplicative_error}

For strictly positive variables,
we may assume a log-multiplicative error model,
%
\begin{equation}
    Y = X \exp(\varepsilon)
    \text{,}
\end{equation}
%
where $\varepsilon$ is the log-scale measurement error,
given by
%
\begin{equation}
    \varepsilon = \log Y - \log X
    \text{.}
\end{equation}
%
The log-scale calibration check error is
%
\begin{equation}
    \varepsilon_d = \log Y_s - \log Z_s = \log\left( \frac{Y_s}{Z_s} \right)
\end{equation}
%
and the log-scale fouling check error is
%
\begin{equation}
    \varepsilon_f = \left( \log Y_b - \log Y_a \right) - \left( \log Z_b - \log Z_a \right)
    = \log\left( \frac{Y_b}{Y_a} \right) - \log\left( \frac{Z_b}{Z_a} \right)
    \text{.}
\end{equation}
%

As before, we model the log-scale error $\varepsilon_t$ as a continuous-time
stochastic process with conditional mean $m(t)$ and variance $v(t)$,
except that the components are additive in logarithmic space
(Equations~\ref{eq:total_mean} and \ref{eq:total_variance}).

Conditioning on the observation $Y=y$ and treating $X$ as fixed,
the conditional distribution of $\log X$ is
%
\begin{equation}
    \log X \mid Y = y
    \sim
    \mathcal{N}\left( \log y - m(t),\, v(t) \right)
    \text{.}
\end{equation}
%

The bias-corrected estimate of $X$ is given in terms of the expectation of $\log X$,
which corresponds to the median of $X$,
%
\begin{equation}
    x_{\operatorname{med}}(t)
    := \exp\left( \mathbb{E}\left[ \log X \mid Y = y \right] \right)
    = y \exp\bigl( -m(t) \bigr)
    \text{.}
\end{equation}
%
Then, the conditional expectation of $X$ can be obtained by
%
\begin{equation}
    x_{\operatorname{mean}}(t)
    := \mathbb{E}\left[ X \mid Y = y \right]
    = y \exp \left( -m(t) + \tfrac{1}{2} v(t) \right)
    \text{,}
\end{equation}
%
where $\tfrac{1}{2}v(t)$ is the standard bias-correction term for lognormal distributions.

Confidence intervals for $X$ are given by
%
\begin{equation}
    \mathrm{CI}
    =
    \left[
        x_{\operatorname{med}}(t)
        \exp\left( -q_{1-\alpha/2} \sqrt{v(t)} \right),
        \;
        x_{\operatorname{med}}(t)
        \exp\left(  q_{1-\alpha/2} \sqrt{v(t)} \right)
    \right]
    \text{,}
\end{equation}
%
where $q_{1-\alpha/2}$ is the $(1-\alpha/2)$ quantile of the standard normal
distribution.

\section{Robust Estimation}
\label{sec:robust_estimation}

The maximum likelihood estimators for $\mu$ and $\sigma^2$ are sensitive to model misspecification,
particularly when check intervals are short
or when measurement precision is mischaracterized.
In such cases, a robust estimator may be preferred.
Here, one could use the median absolute deviation (MAD)
instead of the sample variance.

For the drift rate, the robust estimator is the median of the increments,
%
\begin{equation}
    \tilde{\mu}_k = \operatorname{median}\left\{
        \frac{e_k - \delta_{k-1} e_{k-1}}{t_k - t_{k-1}}
        \,:\, k = 1, 2, \ldots, K
    \right\}
    \text{.}
\end{equation}
%

For the diffusion coefficient, define the standardized absolute residuals
%
\begin{equation}
    \left| r_k^* \right| = \frac{\left| e_k - \delta_{k-1} e_{k-1} - \tilde{\mu}_k (t_k - t_{k-1}) \right|}{\sqrt{t_k - t_{k-1}}}
    \text{.}
\end{equation}
%
The MAD-based robust estimator is then
%
\begin{equation}
    \tilde{\sigma}_k^2
    =
    \left(
        1.4826 \cdot \operatorname{median}\left\{
            \left| r_k^* \right| \,:\, k = 1, 2, \ldots, K
        \right\}
    \right)^2
    \text{,}
\end{equation}
%
where the factor $1.4826 \approx 1/\Phi^{-1}(0.75)$ provides consistency
with the standard deviation under the Gaussian assumption.

\noappendix       %% use this to mark the end of the appendix section. Otherwise the figures might be numbered incorrectly (e.g. 10 instead of 1).

%% Regarding figures and tables in appendices, the following two options are possible depending on your general handling of figures and tables in the manuscript environment:

%% Option 1: If you sorted all figures and tables into the sections of the text, please also sort the appendix figures and appendix tables into the respective appendix sections.
%% They will be correctly named automatically.

%% Option 2: If you put all figures after the reference list, please insert appendix tables and figures after the normal tables and figures.
%% To rename them correctly to A1, A2, etc., please add the following commands in front of them:

\appendixfigures  %% needs to be added in front of appendix figures

\appendixtables   %% needs to be added in front of appendix tables

%% Please add \clearpage between each table and/or figure. Further guidelines on figures and tables can be found below.

\authorcontribution{
    Both authors contributed equally.
} %% this section is mandatory

\competinginterests{
    Authors T.O.H. and G.S. are employees of the U.S. Geological Survey.
    } %% this section is mandatory even if you declare that no competing interests are present

\disclaimer{
    Any use of trade, firm, or product names is for descriptive purposes
    only and does not imply endorsement by the U.S. Government.
} %% optional section


\begin{acknowledgements}
    This work was supported by the
    % National Hydrologic Monitoring Program
    % TODO
    Groundwater and Streamflow Information Program
    of the U.S. Geological Survey.
\end{acknowledgements}




%% REFERENCES

%% The reference list is compiled as follows:

%% \begin{thebibliography}{}
%% 
%% \bibitem[AUTHOR(YEAR)]{LABEL1}
%% REFERENCE 1
%% 
%% \bibitem[AUTHOR(YEAR)]{LABEL2}
%% REFERENCE 2
%% 
%% \end{thebibliography}

%% Since the Copernicus LaTeX package includes the BibTeX style file copernicus.bst,
%% authors experienced with BibTeX only have to include the following two lines:
%%
\bibliographystyle{copernicus}
\bibliography{timeseries_error_model.bib}
%%
%% URLs and DOIs can be entered in your BibTeX file as:
%%
%% URL = {http://www.xyz.org/~jones/idx_g.htm}
%% DOI = {10.5194/xyz}


%% LITERATURE CITATIONS
%%
%% command                        & example result
%% \citet{jones90}|               & Jones et al. (1990)
%% \citep{jones90}|               & (Jones et al., 1990)
%% \citep{jones90,jones93}|       & (Jones et al., 1990, 1993)
%% \citep[p.~32]{jones90}|        & (Jones et al., 1990, p.~32)
%% \citep[e.g.,][]{jones90}|      & (e.g., Jones et al., 1990)
%% \citep[e.g.,][p.~32]{jones90}| & (e.g., Jones et al., 1990, p.~32)
%% \citeauthor{jones90}|          & Jones et al.
%% \citeyear{jones90}|            & 1990



%% FIGURES

%% When figures and tables are placed at the end of the MS (article in one-column style), please add \clearpage
%% between bibliography and first table and/or figure as well as between each table and/or figure.

% The figure files should be labelled correctly with Arabic numerals (e.g. fig01.jpg, fig02.png).


%% ONE-COLUMN FIGURES

%%f
%\begin{figure}[t]
%\includegraphics[width=8.3cm]{FILE NAME}
%\caption{TEXT}
%\end{figure}
%
%%% TWO-COLUMN FIGURES
%
%%f
%\begin{figure*}[t]
%\includegraphics[width=12cm]{FILE NAME}
%\caption{TEXT}
%\end{figure*}
%
%
%%% TABLES
%%%
%%% The different columns must be separated with a & command and should
%%% end with \\ to identify the column break.
%
%%% ONE-COLUMN TABLE
%
%%t
%\begin{table}[t]
%\caption{TEXT}
%\begin{tabular}{column = lcr}
%\tophline
%
%\middlehline
%
%\bottomhline
%\end{tabular}
%\belowtable{} % Table Footnotes
%\end{table}
%
%%% TWO-COLUMN TABLE
%
%%t
%\begin{table*}[t]
%\caption{TEXT}
%\begin{tabular}{column = lcr}
%\tophline
%
%\middlehline
%
%\bottomhline
%\end{tabular}
%\belowtable{} % Table Footnotes
%\end{table*}
%
%%% LANDSCAPE TABLE
%
%%t
%\begin{sidewaystable*}[t]
%\caption{TEXT}
%\begin{tabular}{column = lcr}
%\tophline
%
%\middlehline
%
%\bottomhline
%\end{tabular}
%\belowtable{} % Table Footnotes
%\end{sidewaystable*}
%
%
%%% MATHEMATICAL EXPRESSIONS
%
%%% All papers typeset by Copernicus Publications follow the math typesetting regulations
%%% given by the IUPAC Green Book (IUPAC: Quantities, Units and Symbols in Physical Chemistry,
%%% 2nd Edn., Blackwell Science, available at: http://old.iupac.org/publications/books/gbook/green_book_2ed.pdf, 1993).
%%%
%%% Physical quantities/variables are typeset in italic font (t for time, T for Temperature)
%%% Indices which are not defined are typeset in italic font (x, y, z, a, b, c)
%%% Items/objects which are defined are typeset in roman font (Car A, Car B)
%%% Descriptions/specifications which are defined by itself are typeset in roman font (abs, rel, ref, tot, net, ice)
%%% Abbreviations from 2 letters are typeset in roman font (RH, LAI)
%%% Vectors are identified in bold italic font using \vec{x}
%%% Matrices are identified in bold roman font
%%% Multiplication signs are typeset using the LaTeX commands \times (for vector products, grids, and exponential notations) or \cdot
%%% The character * should not be applied as multiplication sign
%
%
%%% EQUATIONS
%
%%% Single-row equation
%
%\begin{equation}
%
%\end{equation}
%
%%% Multiline equation
%
%\begin{align}
%& 3 + 5 = 8\\
%& 3 + 5 = 8\\
%& 3 + 5 = 8
%\end{align}
%
%
%%% MATRICES
%
%\begin{matrix}
%x & y & z\\
%x & y & z\\
%x & y & z\\
%\end{matrix}
%
%
%%% ALGORITHM
%
%\begin{algorithm}
%\caption{...}
%\label{a1}
%\begin{algorithmic}
%...
%\end{algorithmic}
%\end{algorithm}
%
%
%%% CHEMICAL FORMULAS AND REACTIONS
%
%%% For formulas embedded in the text, please use \chem{}
%
%%% The reaction environment creates labels including the letter R, i.e. (R1), (R2), etc.
%
%\begin{reaction}
%%% \rightarrow should be used for normal (one-way) chemical reactions
%%% \rightleftharpoons should be used for equilibria
%%% \leftrightarrow should be used for resonance structures
%\end{reaction}
%
%
%%% PHYSICAL UNITS
%%%
%%% Please use \unit{} and apply the exponential notation (e.g. 20\,\unit{W\,m^{-2}})


\end{document}
