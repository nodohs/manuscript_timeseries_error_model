
%% Copernicus Publications Manuscript Preparation Template for LaTeX Submissions
%% ---------------------------------
%% This template should be used for copernicus.cls
%% The class file and some style files are bundled in the Copernicus Latex Package, which can be downloaded from the different journal webpages.
%% For further assistance please contact Copernicus Publications at: production@copernicus.org
%% https://publications.copernicus.org/for_authors/manuscript_preparation.html


%% Please use the following documentclass and journal abbreviations for preprints and final revised papers.

%% 2-column papers and preprints
\documentclass[gi, manuscript]{copernicus}



%% Journal abbreviations (please use the same for preprints and final revised papers)


% Advances in Geosciences (adgeo)
% Advances in Radio Science (ars)
% Advances in Science and Research (asr)
% Advances in Statistical Climatology, Meteorology and Oceanography (ascmo)
% Aerosol Research (ar)
% Annales Geophysicae (angeo)
% Archives Animal Breeding (aab)
% Atmospheric Chemistry and Physics (acp)
% Atmospheric Measurement Techniques (amt)
% Biogeosciences (bg)
% Climate of the Past (cp)
% DEUQUA Special Publications (deuquasp)
% Earth Surface Dynamics (esurf)
% Earth System Dynamics (esd)
% Earth System Science Data (essd)
% E&G Quaternary Science Journal (egqsj)
% EGUsphere (egusphere) | This is only for EGUsphere preprints submitted without relation to an EGU journal.
% European Journal of Mineralogy (ejm)
% Geochronology (gchron)
% Geographica Helvetica (gh)
% Geoscience Communication (gc)
% Geoscientific Instrumentation, Methods and Data Systems (gi)
% Geoscientific Model Development (gmd)
% History of Geo- and Space Sciences (hgss)
% Hydrology and Earth System Sciences (hess)
% Journal of Bone and Joint Infection (jbji)
% Journal of Environmentally Compatible Air Transport System (jecats)
% Journal of Micropalaeontology (jm)
% Journal of Sensors and Sensor Systems (jsss)
% Magnetic Resonance (mr)
% Mechanical Sciences (ms)
% Natural Hazards and Earth System Sciences (nhess)
% Nonlinear Processes in Geophysics (npg)
% Ocean Science (os)
% Polarforschung - Journal of the German Society for Polar Research (polf)
% Proceedings of the International Association of Hydrological Sciences (piahs)
% Proceedings of the International Ocean Drilling Programme (piodp)
% Safety of Nuclear Waste Disposal (sand)
% Scientific Drilling (sd)
% SOIL (soil)
% Solid Earth (se)
% State of the Planet (sp)
% The Cryosphere (tc)
% Weather and Climate Dynamics (wcd)
% Web Ecology (we)
% Wind Energy Science (wes)


%% \usepackage commands included in the copernicus.cls:
%\usepackage[german, english]{babel}
%\usepackage{tabularx}
%\usepackage{cancel}
%\usepackage{multirow}
%\usepackage{supertabular}
%\usepackage{algorithmic}
%\usepackage{algorithm}
%\usepackage{amsthm}
%\usepackage{float}
%\usepackage{subfig}
%\usepackage{rotating}


\begin{document}

%\title{Correcting time-series measurements affected by fouling and calibration drift}
\title{Bias correction and confidence intervals for time series with fouling and calibration drift}


% \Author[affil]{given_name}{surname}

\Author[1][thodson@usgs.gov]{Timothy O.}{Hodson} %% correspondence author
\Author[1]{Gregory E.}{Schwarz}

\affil[1]{U.S. Geological Survey Water Resources Mission Area}

%% The [] brackets identify the author with the corresponding affiliation. 1, 2, 3, etc. should be inserted.

%% If an author is deceased, please add \deceased[$Deceased date if applicable$]{$Author number$} (e.g. \deceased[13 November 2015]{2}) at the end of the affiliations. The author number depends on the placement of the author in the author list, e.g. the third author has number 3.


%% If authors contributed equally, please add \equalcontrib{$Author numbers$} (e.g. \equalcontrib{1,3}) at the end of the affiliations. The author number depends on the placement of the author in the author list, e.g. the third author has number 3.




\runningtitle{Fouling and Calibration-drift}

\runningauthor{Hodson and Schwarz}





\received{}
\pubdiscuss{} %% only important for two-stage journals
\revised{}
\accepted{}
\published{}

%% These dates will be inserted by Copernicus Publications during the typesetting process.


\firstpage{1}

\maketitle



\begin{abstract}
    Environmental time-series measurements can accumulate systematic errors
    from fouling and calibration drift.
    Operators therefore routinely check, clean, and recalibrate measurement devices.
    By modeling error growth between checks as a stochastic process,
    we derive bias corrections and confidence intervals for the time series.
\end{abstract}


% \copyrightstatement{TEXT} %% This section is optional and can be used for copyright transfers.


\introduction  %% \introduction[modified heading if necessary]
%
Environmental monitoring often relies on in situ measurements
collected over extended deployment periods.
In practice, these measurements can exhibit systematic errors
from fouling and calibration drift,
as well as irreducible noise.
Consequently, the measurement device must be periodically checked against known references,
cleaned, and recalibrated, which resets the systematic error to zero.
Later, the information from these checks is used to correct
the raw measurements by linearly interpolating errors
between check measurements \citep{Wagner_2006}.
This correction procedure is mathematically equivalent to modeling the error
as a Wiener process conditioned on the check measurements.
Following that assumption, we derive expressions for the conditional mean and variance,
which are equivalent to Kalman filtering and smoothing for this system.
As the checks are also imprecise,
their uncertainty is propagated into the time-series model.
Next, we extend the model to multipoint calibrations
and monitoring networks of similar devices.
The result is a transparent and tractable framework for estimating
fouling and drift corrections with confidence intervals
that is consistent with standard operating procedures for environmental monitoring.


\section{Preliminaries}
%
Let $X$ be the true value of some variable
and let $Y$ be its measured value,
then the measurement error $E$ is the difference
%
\begin{equation}
    E = Y - X
    \label{eq:absolute_error}
    \text{.}
\end{equation}
We assume the measurement error has three independent components---
fouling $E_f$, calibration drift $E_d$,
and measurement noise $E_m$---
that combine additively,
such that the total error is
%
\begin{equation}
    E = E_f + E_d + E_m
    \text{.}
\end{equation}
%
Fouling is any physical, chemical, or biological buildup on the measurement device
causing systematic error.
Under steady-state conditions,
fouling error is assessed by comparing measurements
taken immediately before and after cleaning the device,
%
\begin{equation}
    E_f = Y_b - Y_a
    \label{eq:fouling_error_simple}
    \text{,}
\end{equation}
%
where $Y_b$ and $Y_a$ are the before-and-after measurements, respectively.
Otherwise, a reference device is needed to adjust for any environmental change
during the cleaning process,
%
\begin{equation}
    E_f = (Y_b - Y_a) - (Z_b - Z_a)
    \label{eq:fouling_error}
    \qquad \text{\citep[][but opposite sign]{Wagner_2006},}
\end{equation}
%
where $Z_b$ and $Z_a$ are the before-and-after measurements
from a co-located reference device.
Note \citet{Wagner_2006} flips the sign of the error to equal the correction,
whereas we follow the convention in statistics.

Any systematic bias remaining after cleaning is attributed to calibration drift,
which is assessed by comparing a clean measurement against a known standard.
Let $Z_s$ be the standard value and let $Y_s$ be the clean measurement of $Z_s$,
then the calibration drift error is given by
%
\begin{equation}
    E_d = Y_s - Z_s
    \label{eq:drift_error}
    \qquad \text{\citep[][but opposite sign]{Wagner_2006}.}
\end{equation}
%

Finally, any residual error after cleaning and recalibrating the device
is attributed to measurement noise,
which is modeled as a Gaussian distribution,
%
\begin{equation}
    E_m \sim \mathcal{N}(0 ,\, \sigma_m^2)
    \label{eq:measurement_error}
    \text{,}
\end{equation}
%
where $\sigma_m$ denotes the measurement-device precision,
and the variance of the measurement error is
%
\begin{equation}
    v_m :=
    \operatorname{Var}[E_m] = \sigma_m^2
    \label{eq:measurement_variance}
    \text{.}
\end{equation}
%
In practice, the measurement precision is determined under controlled conditions
and usually reported in the device specifications
(sometimes called the ``accuracy'')
as the $2\sigma_m$ (approximately 95\%) confidence interval. 

Rather than constant, the precision may be heteroscedastic and specified
piecewise over ranges of the measurand. For example, an instrument might be
specified as follows: 0--10\,mg\,L$^{-1}$: $\pm 0.1$\,mg\,L$^{-1}$ or 5\% of the
reading (whichever is greater); 10--30\,mg\,L$^{-1}$: 7\% of the reading.
In that case, the conditional variance is a function of $y$, given by
%
\begin{equation}
    v_m(y)
    :=
    \operatorname{Var}[E_m \mid Y = y]
    =
    \begin{cases}
        \max\left( (0.1/2)^2,\, (0.05 y/2)^2 \right), & 0 \leq y < 10 \\
        (0.07 y/2)^2, & 10 \leq y \leq 30 \\
        \text{undefined}, & \text{otherwise}
    \end{cases}
    \text{,}
\end{equation}
where the division by 2 converts the $2\sigma$ specification to $\sigma$.

\section{Wiener-Process Error Model}

Let $E(t)$ denote the measurement error at time $t$, 
and let $e(t)$ denote its realization.
Observations are available for a finite set of times
when fouling or calibration are checked,
$t_0 < t_1 < \cdots < t_K$.
For notational convenience, we define
$E_k := E(t_k)$
and
$e_k := e(t_k)$.
Thus, $t_k - t_{k-1}$ denotes the time elapsed between successive error
observations $e_{k-1}$ and $e_k$.
Throughout, the index $k$ will be substituted with $f$ and $d$
to denote fouling and calibration-drift checks, respectively.

We assume the fouling or calibration-drift error evolve as a continuous-time
stochastic process, defined by
%
\begin{equation}
    E_t = \delta_{k-1} E_{k-1} + \mu (t - t_{k-1}) + \sigma W_t
    \text{,}
\end{equation}
%
where $\delta_{k-1}$ indicates whether the error process was reset during the previous check
(because the device was cleaned or recalibrated),
%
\begin{equation}
    \delta_{k-1} =
    \begin{cases}
        0, & \text{reset at } t_{k-1} \\
        1, & \text{otherwise}
    \end{cases}
    \text{,}
\end{equation}
%
%$e_{k-1}$ is the error at time $t_{k-1}$,
%$\mathcal{N}$ is the normal distribution
%specified by its mean and variance,
$\mu$ is the ``drift'' rate (mean change per unit time),
$\sigma$ is the ``diffusion'' rate,
and $W_t$ is a standard Wiener process,
%
\begin{equation}
    W_t \sim \mathcal{N}(0,\, t - t_{k-1})
    \qquad \text{\citep{Taylor_1998}.}
\end{equation}
%
The drift term $\mu$ represents the mean rate of change
and is important for devices that gradually lose responsiveness due
to the degradation of their components over time,
like the chemical degradation of reagents in a colorimetric analyzer.

The timeseries $E_t$ is subdivided into intervals that are bracketed by check measurements.
An interval bracketed by a single check is called ``open.''
For example, the interval after the last check ($t > t_K$) is open until
the next check occurs sometime in the future ($t_{K+1}$).
In contrast, a ``closed'' interval is bracketed by checks at both ends ( $ t \in [t_{k-1}, t_k]$ ).
In real-time monitoring, the present time is always within an open interval.
Once the next check is recorded, that interval is closed and a new open interval begins.

For an open interval,
the conditional distribution of the fouling or calibration-drift error is
%
\begin{equation}
    p(e_t \mid E_{k-1} = e_{k-1})
    =
    \mathcal{N}(m_k^o(t),\, v_k^o(t))
    \text{,} \qquad
    t > t_{k-1}
    %\mathcal{N}(\delta_{k-1}  e_{k-1} + \mu t ,\, \sigma^2 t)
\end{equation}
%
with conditional mean and variance given by
%
\begin{align}
    m_k^o(t) &:=
        \mathbb{E}\left[E_t \mid E_{k-1} = e_{k-1} \right]
        = \delta_{k-1} e_{k-1} + \mu (t - t_{k-1})
        \qquad \text{and}
        \label{eq:wiener_mean}\\
    v_k^o(t) &:=
        \operatorname{Var}\left[E_t \mid E_{k-1} = e_{k-1} \right]
        = \sigma^2 (t - t_{k-1})
        \label{eq:wiener_variance}
    \text{.}
\end{align}
%

Upon observing the error at time $t_k$,
the conditional distribution of $E_t$ is updated and the interval is closed.
A Wiener process conditioned to start and end at specified values
is called a Brownian bridge \citep{Taylor_1998}.
To simplify notation, let $\lambda_k(t)$ be the linear-interpolation weight,
%
\begin{equation}
    \lambda_k(t) = \frac{t - t_{k-1}}{t_k - t_{k-1}}
    \text{,} \qquad
    t \in [t_{k-1}, t_k]
    \text{.}
\end{equation}
%
The conditional distribution of the bridge at time $t$ is
%
\begin{equation}
    p(e_t \mid E_{t_{k-1}} = e_{k-1},\, E_{t_k} = e_k)
    =
    \mathcal{N} \left(
        m^c(t),\, v^c(t)
        %\delta_{k-1} \left( 1 - \lambda(t) \right)e_{k-1}
        %+ 
        %\lambda(t) e_k,\;
        %\sigma^2 (1 - \lambda(t)) \lambda(t) (t_k - t_{k-1})
    \right)
    \text{,} \qquad
    t \in \left[t_{k-1},\, t_k \right]
\end{equation}
%
with conditional mean and variance given by
%
\begin{align}
    m_k^c(t) &:=
        \mathbb{E}\left[ E_t \mid E_{t_{k-1}} = e_{k-1} ,\, E_{t_k} = e_k \right]
        =
        \delta_{k-1} \left( 1 - \lambda_k(t) \right) e_{k-1} + \lambda_k(t) e_k
        \qquad \text{and}
        \label{eq:bridge_mean} \\
    v_k^c(t) &:=
    \operatorname{Var} \left[ E_t \mid E_{t_{k-1}} = e_{k-1} ,\, E_{t_k} = e_k \right]
        = \sigma^2
        \underbrace{
            \frac{(t - t_{k-1})(t_{k} - t)}{t_{k} - t_{k-1}} 
        }_{\text{bridge factor}}
        % = \sigma^2 (1 - \lambda_k(t)) (t - t_{k-1})
        \label{eq:bridge_variance}
    \text{.}
\end{align}
%
Extending $\lambda_k(t)$ as
%
\begin{equation}
    \lambda^+_k(t) :=
    \begin{cases}
        1, & t < t_{1} \\
        \frac{t - t_{k-1}}{t_k - t_{k-1}}, & t \in [t_{k-1}, t_k] \\
        0, & t > t_{K}
    \end{cases}
    \text{,}
\end{equation}
%
so $\lambda^+_k(t) = 1$ or $0$ for open intervals,
and letting $\phi^+_k(t)$ be the bridge factor, extended by
%
\begin{equation}
    \phi^+_k(t) :=
    \begin{cases}
        (t_{1} - t), & t < t_{1} \\
        \frac{(t - t_{k-1})(t_{k} - t)}{(t_{k} - t_{k-1})},
        & t \in [t_{k-1}, t_{k}] \\
        (t - t_{K}), & t > t_{K} 
    \end{cases}
    \text{,}
\end{equation}
and letting $\zeta^+_k(t)$ be a drift factor, extended by
%
\begin{equation}
    \zeta^+_k(t) :=
    \begin{cases}
        (t_{1} - t), & t < t_{1} \\
        0 , & t \in [t_{k-1}, t_{k}] \\
        (t - t_{K}), & t > t_{K}
    \end{cases}
    \text{,}
\end{equation}
so the drift term is set to zero for closed intervals.
%
Then conditional mean and variance can be written without superscripts
($m^o$, $m^c$, etc.) as
%
\begin{align}
    m_k(t)  &=  \delta_{k-1} \left( 1 - \lambda^+_k(t) \right)e_{k-1}
                + \lambda^+_k(t) e_k
                + \mu \zeta^+_k(t)
    \label{eq:unified_wiener_mean}
    \qquad \text{and} \qquad \\
    v_k(t) &= \sigma^2 \phi^+_k(t)
    \label{eq:unified_wiener_variance}
    \text{.}
\end{align}

\section{Check-Measurement Uncertainty}

Noise in the check measurements contributes additional uncertainty to the error process.
After reviewing the general approach for propagating check-measurement uncertainty,
we derive formulas for fouling and calibration-drift checks.

For a bridge with uncertain endpoints,
the endpoint expectations and variances are defined as
%
\begin{equation}
    e_{k-1} := \mathbb{E}\left[ E_{k-1}\right]
    \text{,} \qquad
    e_{k} := \mathbb{E}\left[ E_{k}\right]
    \text{,} \qquad
    v_{k-1} := \operatorname{Var}\left[ E_{k-1}\right]
    \text{,} \qquad
    v_k := \operatorname{Var}\left[ E_{k}\right]
\end{equation}
%
Given the Brownian-bridge,
%
\begin{equation}
    E_t = (1 - \lambda^+_k(t)) E_{k-1} + \lambda^+_k(t) E_k + \sigma W_t
    \text{,} \qquad
    W_t \sim \mathcal{N}\left(0, \phi^+_k(t)\right)
    \text{,}
\end{equation}
%
and applying linearity of the expectation,
%
\begin{align}
    \mathbb{E}[E_t] &= (1 - \lambda^+_k(t)) \, \mathbb{E}[E_{k-1}]
                        + \lambda^+_k(t) \, \mathbb{E}[E_k] + \mathbb{E}[W_t]
                        \notag \\
                    &= (1 - \lambda^+_k(t)) e_{k-1} + \lambda^+_k(t) e_k  
    \text{,}
\end{align}
%
shows that the conditional mean remains a linear interpolation between the endpoint means,
regardless of the endpoint variances.

Assuming the endpoint errors are conditionally independent given the check measurements $\mathcal{C}$,
%
\begin{equation}
    \operatorname{Cov}\left[ E_{k-1},\, E_k \mid \mathcal{C} \right] = 0
    \text{,}
\end{equation}
%
and independent of $W_t$,
the combined variance is obtained by applying the variance formula for a linear combination,
%
\begin{align}
    \operatorname{Var}[E_t]
        &= \operatorname{Var}[
            (1 - \lambda^+_k(t)) \, E_{k-1}
            +
            \lambda^+_k(t) \, E_{k}
            ]
            + \operatorname{Var}[W_t]
            \notag \\
        &=  \underbrace{
            (1 - \lambda^+_k(t))^2 \, v_{k-1}
            + \lambda^+_k(t)^2 \, v_k
            }_{\tilde{v}_k(t)}
            + 
            \underbrace{\sigma^2 \phi^+_k(t)}_{v_k(t)}
    \label{eq:variance_uncertain_bridge}
    \text{,}
\end{align}
%
where $\tilde{v}_k(t)$ is the contribution from check-measurement uncertainty
(Figure~\ref{fig:variance_between_uncertain_checkpoints}).

\begin{figure*}[t]
\includegraphics[width=12cm]{
    figures/variance_between_uncertain_checkpoints.pdf
}
\caption{
    Error variance during open and closed-interval intervals with uncertain check measurements.
    where the diffusion rate is
    $\sigma^2 = 0.02$,
    and the check variances are
    $v(t=0) = 0.04$
    and
    $v(t=10) = 0.09$.
}
\label{fig:variance_between_uncertain_checkpoints}
\end{figure*}
%


If the endpoint errors have equal variance,
$v_{k-1} = v_{k}$,
the conditional variance simplifies to
%
\begin{equation}
    \tilde{v}_k(t) = \left(
                        (1 - \lambda^+_k(t))^2 + \lambda^+_k(t)^2
                      \right) \, v_{k}
    \label{eq:check_constant_variance}
    \text{.}
\end{equation}
%
For open intervals, this reduces to
$ \tilde{v}^o_k(t) = v_{k-1}$,
which is constant in time.

The next sections derive variance formula for the
fouling and calibration-drift checks,
$v_{f}$ and $v_{d}$,
to use in Equation~\ref{eq:check_constant_variance}.
Later, we introduce multipoint calibrations,
where the check variances often differ;
so we will use Equation~\ref{eq:variance_uncertain_bridge} for that case.

\subsection{Fouling-Check Uncertainty}

For a fouling check, the measurement $m$ and reference $r$
devices have known precisions
and normally distributed, independent errors,
\begin{equation}
    E^{m} \sim \mathcal{N}\left(0,\, v_m(y) \right)
    \text{,} \qquad
    E^{r} \sim \mathcal{N}\left(0,\, v_r(z) \right)
    \text{,} \qquad
    E^{m} \perp E^{r}
    \text{.}
\end{equation}
%
Let $Y$ and $Z$ be the principal and reference measurements,
and let $X^m$ and $X^r$ be their respective true values,
%
\begin{equation}
    Y = X^m + E^{m}
    \text{,} \qquad
    Z = X^r + E^{r}
    \text{.}
\end{equation}
%
The superscripts are necessary because $X$
depends on the calibration of either device.
Substituting back into the fouling error (Equation~\ref{eq:fouling_error})
gives
%
\begin{align}
    E_f &= (Y_b - Y_a) - (Z_b - Z_a) \notag \\
        &= \left((X^m_b + E^m_{b}) - (X^m_a + E^m_{a})\right)
           - \left((X^r_b + E^r_{b}) - (X^r_a + E^r_{a})\right) \notag \\
        &= (E^m_{b} - E^m_{a}) - (E^r_{b} - E^r_{a})
    \text{.}
\end{align}
%
By independence ($E^{m} \perp E^{r}$), the variance of the fouling check is
\begin{align}
    \operatorname{Var}[E_f]
        &= \operatorname{Var}[E^m_b - E^m_a]
          + \operatorname{Var}[E^r_b - E^r_a]
    \text{.}
\end{align}
%
If the measurement device uses internal filtering
(e.g., time averaging),
we assume its errors are perfectly correlated during the check,
such that the before-and-after errors cancel and the check variance is zero,
%
\begin{equation}
    \operatorname{Corr} \left[ E^{m}_{a},\, E^{m}_{b} \right] = 1
    \quad \Rightarrow \quad
    \operatorname{Var}[E_f] =
    \operatorname{Var} \left[E^m_b - E^m_a\right] = 0
    \text{.}
\end{equation}
%
Similarly, for the reference device, if it also uses filtering.
Otherwise, we assume its errors are independent,
and the check variance is given by
%
\begin{equation}
    \operatorname{Corr}\left[E^{r}_a,\, E^{r}_b\right] = 0
    \quad \Rightarrow \quad
    \operatorname{Var} \left[E_f\right]
    = \operatorname{Var} \left[E^r_b\right] + \operatorname{Var}\left[E^r_a\right]
    = v_r(z_b) + v_r(z_a)
    % = \sigma^2_r + \sigma^2_r = 2 \sigma^2_r
    \text{.}
\end{equation}
%
Thus, under constant precision, the fouling-check variance is either zero
(if the reference-device errors are perfectly correlated during the check)
or $2\sigma_r^2$ (if the errors are independent).

\subsection{Calibration-Check Uncertainty}

For a calibration check, 
the standard and measurement have known precisions $\sigma_s$ and $\sigma_m$
and independent errors,
%
\begin{equation}
    E^{s} \sim \mathcal{N}\left(0,\, \sigma^2_s \right)
    \text{,} \qquad
    E^{m} \sim \mathcal{N}\left(0,\, v_m(z_s) \right)
    \text{,} \qquad
    E^{s} \perp E^{m}
    \text{.}
\end{equation}
%
Let $Y_s$ and $Z_s$ be the measurement and the standard,
and let $X_s$ be the true value of the standard, such that
%
\begin{equation}
    Y_s = X_s + E^{m}
    \text{,} \qquad \text{and} \qquad
    Z_s = X_s + E^{s}
    \text{.}
\end{equation}
%
Substituting back into the drift error
(Equation~\ref{eq:drift_error}) gives
%
\begin{equation}
    E_d = (X_s + E^m) - (X_s + E^s)
    \text{,}
\end{equation}
%
and by independence ($E^{m} \perp E^{s}$), the variance of the drift check is
%
\begin{equation}
    \operatorname{Var}[E_d]
        = \operatorname{Var}[E^m] + \operatorname{Var}[E^s]
        = v_m(z_s) + \sigma^2_s
    \text{.}
\end{equation}
%

\section{Multipoint Check}

In a multipoint check, a device is checked against multiple reference values
that should span the expected range of the measurand.
To distinguish these from the reference device used in fouling checks,
we call them ``standards,''
although in some cases ``reference value'' may be more appropriate.
Multipoint checks are essential when the device response is nonlinear
and the error magnitude varies across its range.
For example, a pH meter might be checked against standards at pH 4, 7, and 10,
then the drift correction interpolates errors at these check points
\citep{Wagner_2006}.

The previous sections showed how single-point checks form a Gaussian bridge through time.
The multipoint check extends the same basic idea,
using the check measurements as boundary conditions for the bridge.
In this case, the boundary condition is not a point but a second bridge between the standards in the $y$ dimension.

Let $z_s$ denote the standard values for $s = 1, 2, \ldots, S$,
let $E_{k,s} := E(t_k,\, z_s)$ be the error at time $t_k$ and standard $z_s$,
and let $e_{k,s}$ be its realization.
Given check-measurement errors $e_{k,s-1}$ and $e_{k,s}$,
the conditional mean of the $y$-bridge at time $t_k$ is given by
%
\begin{align}
m_{k}(y) &:= \mathbb{E} \left[
                            E_{k,y}
                            \mid
                            E_{k,s-1} = e_{k,s-1},\,
                            E_{k,s} = e_{k,s}
                           \right]
            \notag \\
            &= (1 - \lambda^+_s(y)) \, \mathbb{E} [E_{k,s-1}]
             + \lambda^+_s(y) \, \mathbb{E} [E_{k,s}] \notag \\
            &= (1 - \lambda^+_s(y)) e_{k,s-1}
             + \lambda^+_s(y) e_{k,s}
\label{eq:y_bridge_mean}
\text{,}
\end{align}
%
which linearly interpolates between the standard values.
Unlike the time-bridge, the $y$-bridge has no drift or reset terms ($\delta$, $\mu$).
The conditional variance of the $y$-bridge at time $t_k$ is
%
\begin{align}
    v_{k}(y)    &:= \operatorname{Var} \left[
                            E_{k,y}
                            \mid
                            E_{k,s-1} = e_{k,s-1},\,
                            E_{k,s} = e_{k,s}
                            \right] \notag \\
                &= (1 - \lambda^+_s(y))^2 \, \operatorname{Var}[E_{k,s-1}]
                    + \lambda^+_s(y)^2 \, \operatorname{Var}[E_{k,s}]
                    + \eta^2 \phi^+_{s}(y)
    \label{eq:y_bridge_variance}
    %\text{(Equation \ref{eq:variance_uncertain_bridge} adapted for the $y$ dimension),}
\end{align}
%
where $\eta$ is the diffusion coefficient in the $y$ dimension,
and $\operatorname{Var}[E_{k,s-1}]$ and $\operatorname{Var}[E_{k,s}]$
are the variances of the checks against standards $s-1$ and $s$.

Under this boundary condition,
the conditional mean is a bilinear interpolation
between four check points,
$e_{k-1,s-1}$, $e_{k-1,s}$, $e_{k,s-1}$, and $e_{k,s}$.
Let $\mathcal{C}$ denote the set of four check errors,
%
\begin{equation}
    \mathcal{C} := \{E_{k-1,s-1} = e_{k-1,s-1},\, E_{k-1,s} = e_{k-1,s},\, 
    E_{k,s-1} = e_{k,s-1},\, E_{k,s} = e_{k,s}\}
    \text{,}
\end{equation}
then the conditional mean is given by
%
\begin{align}
    m_k(t, y)
    &:=
    \mathbb{E}\left[
        E_{t,y}\,\mid\,
        \mathcal{C}
        \right] \notag \\
    &=
    (1-\lambda^+_k(t),\;\lambda^+_k(t))
    \begin{bmatrix}
        \delta_{k-1} e_{k-1,s-1} & \delta_{k-1} e_{k-1,s} \\
        e_{k,s-1}   & e_{k,s}
    \end{bmatrix}
    \begin{bmatrix}
        1-\lambda^+_s(y) \\
        \lambda^+_s(y)
    \end{bmatrix}
    \label{eq:multi_check_mean}
    \text{.}
\end{align}
%
Expanding the right-hand side yields
%
\begin{align}
    m_k(t,y)
    &=
    (1 - \lambda^+_k(t))(1 - \lambda^+_s(y)) \delta_{k-1} e_{k-1,s-1} \notag \\
    &\quad
    + (1 - \lambda^+_k(t)) \lambda^+_s(y) \delta_{k-1} e_{k-1,s} \notag \\
    &\quad
    + \lambda^+_k(t)(1 - \lambda^+_s(y)) e_{k,s-1} \notag \\
    &\quad
    + \lambda^+_k(t) \lambda^+_s(y) e_{k,s}
    \text{,}
\end{align}
%
which satisfies the conditions for the four check points,
%
\begin{align}
    m_k(t_{k-1}, y_{s-1}) &=  \delta_{k-1}e_{k-1,s-1} \text{,} \notag \\
    m_k(t_{k-1}, y_{s})   &=  \delta_{k-1}e_{k-1,s} \text{,} \notag \\
    m_k(t_{k}, y_{s-1})   &=  e_{k,s-1} \text{,} \notag \\
    m_k(t_{k}, y_{s})     &=  e_{k,s} \text{.}
\end{align}
%
To derive the conditional variance,
substitute the variance of the $y$-bridge
(Equation~\ref{eq:y_bridge_variance})
into the equation for a time bridge with uncertain endpoints
(Equation~\ref{eq:variance_uncertain_bridge})
%
\begin{align}
    \operatorname{Var}\left[
        E_{t,y}\,\mid\,
        \mathcal{C}
        \right]
    =  
    \underbrace{
        (1 - \lambda^+_k(t))^2 v_{k-1}(y)
        + \lambda^+_k(t)^2 v_k(y)
    }_{\tilde{v}_k(t,y)}
    + 
    \underbrace{\sigma^2_k \phi^+_k(t)}_{v_k(t)}
    \text{.}
\end{align}
%
Expanding the right-hand side yields
%
\begin{align}
    \tilde{v}_k(t,y) &:=
        \left((1 - \lambda^+_k(t))^2,\; \lambda^+_k(t)^2\right)
        \begin{bmatrix}
            v_{k-1,s-1} & v_{k-1,s} \\
            v_{k,s-1}   & v_{k,s}
        \end{bmatrix}
        \begin{bmatrix}
            (1 - \lambda^+_s(y))^2 \\
            \lambda^+_s(y)^2
        \end{bmatrix}
        \notag \\
        &\quad
        + \left((1 - \lambda^+_k(t))^2 + \lambda^+_k(t)^2\right)\eta^2 \phi^+_s(y)
        + \sigma^2_k \phi^+_k(t)
    \label{eq:multi_check_variance}
    \text{,}
\end{align}
%
which satisfies the four boundary conditions
%
\begin{align}
    \tilde{v}_k(t, y_{s-1})   &=    (1 - \lambda^+_k(t))^2 v_{k-1,s-1}
                                    + \lambda^+_k(t)^2 v_{k,s-1}
                                    + \sigma^2_k \phi^+_k(t)
                                    \text{,} \notag \\
    \tilde{v}_k(t, y_{s})     &=    (1 - \lambda^+_k(t))^2 v_{k-1,s}
                                    + \lambda^+_k(t)^2 v_{k,s}
                                    + \sigma^2_k \phi^+_k(t)
                                    \text{,} \notag \\
    \tilde{v}_k(t_{k-1}, y)   &=    (1 - \lambda^+_s(y))^2 v_{k-1,s-1}
                                    + \lambda^+_s(y)^2 v_{k-1,s}
                                    + \eta^2 \phi^+_s(y)
                                    \text{,} \notag \\
    \tilde{v}_k(t_{k}, y)     &=    (1 - \lambda^+_s(y))^2 v_{k,s-1} 
                                    + \lambda^+_s(y)^2 v_{k,s}
                                    + \eta^2 \phi^+_s(y)
    \text{,}
\end{align}
%
each being a one-dimensional Gaussian bridge.

Another application of a multipoint check is to apply a percentage correction
using a single reference value.
This is common in fouling checks where the only available reference is the post-cleaning value.
For consistency with the multipoint notation,
let this check value be at $z_s$ with error $e_s$ and variance $v_s$.
Here, we may apply a percentage or absolute correction depending on the device characteristics.
For an absolute correction,
we apply the appropriate formula from previous sections
(Equations~\ref{eq:unified_wiener_mean} and \ref{eq:variance_uncertain_bridge}).
For a percentage correction,
we apply the multipoint formula assuming that $e_0=0$ and $v_0=0$ at $z_0 = 0$,
such that the error scales linearly with the measurand.
Then, the interpolation weight simplifies to $\lambda^+_s(y) = y/z_s$
and the bridge factor to $\phi^+_s(y) = y(z_s - y)/z_s$.
Substituting back into the conditional mean (Equation~\ref{eq:multi_check_mean}) gives
%
\begin{align}
    m_k(t,y) &= \frac{y}{z_s} \left[ (1 - \lambda^+_k(t)) \delta_{k-1} e_{k-1,s} + \lambda^+_k(t) e_{k,s} \right]
    \text{,}
\end{align}
%
and the conditional variance (Equation~\ref{eq:multi_check_variance}) gives 
\begin{equation}
    v_k(t,y) = \left(\frac{y}{z_s}\right)^2 v_s + \eta^2 \frac{y(z_s - y)}{z_s}
    \text{.}
\end{equation}
%

\section{Combined Error}

Recall that the fouling and calibration-drift variance has two components:
%
\begin{equation}
    \dot{v}_k := \operatorname{Var}\!\left[E_k\right] = v_k(t) + \tilde v_k(t)
    \text{,}
\end{equation}
%
where $v_k(t)$ is the variance from the error-process evolution 
and $\tilde v_k(t)$ is the additional variance from check-measurement uncertainty
(Equations~\ref{eq:check_constant_variance} or \ref{eq:multi_check_variance}).

The combined conditional mean and variance are given by the sum of the individual components,
%
\begin{align}
m(t ,\, y) &=
    m_{f}(t ,\, y)
    +
    m_{d}(t ,\, y)
    \label{eq:total_mean}
    \qquad \text{and} \\
v(t ,\, y) &=
    \dot{v}_{f}(t ,\, y)
    +
    \dot{v}_{d}(t ,\, y)
    +
    v_{m}(y)
    \label{eq:total_variance}
    \text{,}
\end{align}
%
where $f$, $d$, and $m$ denote the fouling, drift, and measurement-noise components, respectively.
By definition, the mean of the measurement noise is zero ($m_m = 0$),
and the other means are given by Equations~
\ref{eq:unified_wiener_mean}
and
\ref{eq:multi_check_mean};
and the variances by Equations
\ref{eq:measurement_error},
\ref{eq:unified_wiener_variance},
\ref{eq:variance_uncertain_bridge},
and
\ref{eq:multi_check_variance}.

\section{Corrected Measurement and Confidence Interval}

Recall that $m(t ,\, y)$ and $v(t ,\, y)$ give the expectation and variance of the error $E$ at $(t,y)$.
For the measurement model $Y = X + E$,
the bias-corrected estimate of the measurand $X$ is given by
%
\begin{equation}
    x(t ,\, y) = y - m(t ,\, y)
    \text{.}
\end{equation}
%
Since errors are conditionally Gaussian,
the approximate confidence interval is
%
\begin{equation}
    \mathrm{CI} = x(t ,\, y) \pm q_{1-\alpha/2}\sqrt{v(t ,\, y)}
    \text{,}
\end{equation}
%
where $q_{1-\alpha/2}$ is the $(1-\alpha/2)$ quantile
of the standard normal distribution.

\section{Parameter Estimation}

Let $r_k$ be the ``one-step-transition'' residual 
%
\begin{equation}
    r_k = e_k - m^o_k(t_k)
\end{equation}
and let $u_k$ be its conditional variance   
\begin{equation}
    u_k = v^o_k(t_k) + v_{k-1} + v_{k}
    \text{,}
\end{equation}
%
where $v_{k-1}$ and $v_{k}$ are the variances of the previous and current checks.

The drift rate $\mu_k$ and diffusion coefficient $\sigma_k$
are estimated by maximizing the Gaussian log-likelihood function
(omitting the additive constants),
%
\begin{equation}
    \ln \mathcal{L} (\mu_k ,\, \sigma_k^2)
    =
    - \frac{1}{2} \sum_{k=1}^{K}
    \left[
        \log \left( u_k \right)
        +
        \frac{r_k^2}{u_k}
    \right]
    \text{.}
\end{equation}
%

For the special case of no observational uncertainty,
$v_k = v_r \equiv 0$,
the drift and diffusion parameters have closed-form maximum-likelihood estimators,
%
\begin{equation}
    \hat{\mu}_k = \frac{\sum_{k=1}^{K} e_k - \delta_{k-1} e_{k-1}}
                       {\sum_{k=1}^{K} (t_k - t_{k-1})}
\end{equation}
%
and
%
\begin{equation}
    \hat{\sigma}_k^2 = \frac{1}{K}{\sum_{k=1}^{K} \frac{r_k^2}{t_k - t_{k-1}}}
    \text{;}
\end{equation}
%
however, these become ill-conditioned when measurement uncertainty is significant,
because the denominator may approach zero while the residual term in the numerator does not.

For the multipoint check,
the residual and its variance are given by
%
\begin{align}
    r_{k,s} &= e_{k,s} - e_{k,s-1} \\
    u_{k,s} &= \eta^2 (z_{k,s} - z_{k,s-1})
                + v_{k,s} + v_{k,s-1}
    \text{,}
\end{align}
%
and the diffusion-rate parameter $\eta$
is estimated by maximizing the log-likelihood (omitting constants),
%
\begin{equation}
    \ln \mathcal{L} (\eta^2)
    =
    %-\frac{K_i}{2} \ln(2 \pi)
    - \frac{1}{2} \sum_{k=1}^{K} \sum_{s=2}^{S_k}
    \left[
        \log \left( u_{k,s} \right)
        +
        \frac{r_{k,s}^2}{u_{k,s}}
    \right]
    \text{,}
\end{equation}
%
where $S_k$ is the number of points in the $k$-th check.
%This is necessary because in practice,
%only some of the standards may be checked at a time.

Assuming no observational uncertainty,
$v_s = v_{s-1} \equiv 0$,
the maximum-likelihood estimator for $\eta$ is
%
\begin{equation}
    \hat{\eta}^2 =
    \frac{1}{N}
    \sum_{k=1}^{K} \sum_{s=2}^{S_k}
    \frac{\left(e_{k,s} - e_{k,s-1}\right)^2}
         {z_s - z_{s-1}}
    \text{,}
\end{equation}
%
where $N = \sum_{k=1}^{K} (S_k - 1)$.
This is better conditioned than $\hat{\sigma}$,
because calibration standards are typically separated in the $y$-dimension,
so the denominator never approaches zero.

\section{Monitoring Networks}

After deploying a measurement device,
several checks may be required before the drift and diffusion
parameters can be estimated reliably.
During this initial period, the estimates can be stabilized
by pooling information across similar devices in a monitoring network.
We propose using an empirical-Bayes approach with conjugate priors to obtain a closed-form shrinkage weight
that shrinks the individual estimate toward the network (population) mean.

Let $s_j$ be the diffusion estimate for device $j \in \{1,\ldots,J\}$,
and let $K_j$ be the number of checks for that particular device.

We assume an inverse-gamma prior on the diffusion rate,
$\sigma_j^2 \sim \operatorname{IG}(\kappa, \theta)$,
where $\kappa>0$ and $\theta>0$ are hyperparameters to be estimated from the data.
The prior has mean $\mathbb{E}[\sigma^2]=\theta/(\kappa-1)$ for $\kappa>1$.
Integrating out $\sigma_j^2$ gives the marginal log-likelihood for $(\kappa,\theta)$,
which, omitting additive constants, is
%
\begin{equation}
    \ln \mathcal{L} (s \mid \kappa,\theta)
    =
    \sum_{j=1}^{J}
    \left(
        \kappa\ln(\theta)
        + \ln \Gamma\left(\kappa + \frac{K_j}{2}\right)
        - \ln \Gamma(\kappa)
        - \left(\kappa + \frac{K_j}{2}\right)\ln\left(\theta + \frac{K_j s_j^2}{2}\right)
    \right)
    \text{.}
    \label{eq:hierarchical_likelihood}
\end{equation}
%
Maximizing Equation~\ref{eq:hierarchical_likelihood} yields empirical-Bayes estimates $\hat{\kappa}$ and $\hat{\theta}$.
The empirical-Bayes shrinkage estimate is
%
\begin{equation}
    \hat{\sigma}_j^2
    =
    \hat{w}_j\,s_j^2 + (1-\hat{w}_j)\,\bar{\hat{\sigma}}^2
    \text{,}
\end{equation}
%
where
\begin{equation}
    \bar{\hat{\sigma}}^2 := \hat{\theta}/(\hat{\kappa}-1)
    \text{,} \qquad
    \hat{w}_j := K_j/(K_j + 2(\hat{\kappa}-1))
    \text{.}
\end{equation}
%

For the drift rate, we assume a Gaussian prior $\mu_j \sim \mathcal{N}(\nu, \tau^2)$,
where $\nu$ and $\tau^2$ are population hyperparameters
with maximum-likelihood estimators
%
\begin{equation}
    \hat{\nu} = \frac{\sum_{j=1}^{J} K_j \hat{\mu}_j}{\sum_{j=1}^{J} K_j}
    \text{,} \qquad
    \hat{\tau}^2 = \frac{\sum_{j=1}^{J} K_j (\hat{\mu}_j - \hat{\nu})^2}{\sum_{j=1}^{J} K_j}
    \text{.}
\end{equation}
%
The empirical-Bayes shrinkage estimate is
%
\begin{equation}
    \tilde{\mu}_j
    =
    \hat{u}_j\,\hat{\mu}_j + (1-\hat{u}_j)\,\hat{\nu}
    \text{,}
\end{equation}
%
where the shrinkage weight follows from the conjugate Gaussian prior,
%
\begin{equation}
    \hat{u}_j := K_j\hat{\tau}^2/(K_j\hat{\tau}^2 + \hat{\sigma}_j^2)
    \text{.}
\end{equation}
%

In practice, per-device estimates for
$(\hat{\sigma}_j^2$, $\hat{w}_j$, $\tilde{\mu}_j$, and $\hat{u}_j)$
can be recomputed after each check,
whereas the population hyperparameters $(\hat{\kappa},\hat{\theta},\hat{\nu},\hat{\tau}^2)$
may be updated less frequently.
The main limitation of our Empirical-Bayes approach is that
the inverse-gamma prior is known to be suboptimal \citep{Gelman_2006}.
Nevertheless, its closed-form shrinkage estimator is advantageous for large, real-time monitoring networks.
Otherwise, each new check measurement incurs a significant computational cost.

\section{Data Requirements}

Implementing the bias-correction framework requires the following information.
%
\begin{enumerate}
    \item The time of each cleaning and recalibration (when the error process resets).
    \item The time and value of each check measurement,
    including any associated reference measurements and standard values.
    \item The precision for the measurement device and any reference devices or standards.
    \item The device type (make and model) for grouping like devices in hierarchical estimation.
    \item A unique device identifier for tracking devices across multiple redeployments.
\end{enumerate}
%

\conclusions  %% \conclusions[modified heading if necessary]

Environmental timeseries are routinely affected by fouling and calibration drift,
which cause systematic bias.
In practice, the measurement device must be periodically checked
against known references, cleaned, and recalibrated.
Later, these check measurements are used to correct the timeseries
by interpolating between the errors observed at each check.
Here, we formalize this correction procedure as a stochastic model
to derive confidence intervals for the corrected timeseries.
We also extend the model to multipoint calibrations
and to pooling information across monitoring networks of similar devices.

%% The following commands are for the statements about the availability of data sets and/or software code corresponding to the manuscript.
%% It is strongly recommended to make use of these sections in case data sets and/or software code have been part of your research the article is based on.

% \codeavailability{TEXT} %% use this section when having only software code available
% \dataavailability{TEXT} %% use this section when having only data sets available
% \codedataavailability{TEXT} %% use this section when having data sets and software code available
% \sampleavailability{TEXT} %% use this section when having geoscientific samples available
% \videosupplement{TEXT} %% use this section when having video supplements available


% \appendix

% \noappendix       %% use this to mark the end of the appendix section. Otherwise the figures might be numbered incorrectly (e.g. 10 instead of 1).

%% Regarding figures and tables in appendices, the following two options are possible depending on your general handling of figures and tables in the manuscript environment:

%% Option 1: If you sorted all figures and tables into the sections of the text, please also sort the appendix figures and appendix tables into the respective appendix sections.
%% They will be correctly named automatically.

%% Option 2: If you put all figures after the reference list, please insert appendix tables and figures after the normal tables and figures.
%% To rename them correctly to A1, A2, etc., please add the following commands in front of them:

% \appendixfigures  %% needs to be added in front of appendix figures

% \appendixtables   %% needs to be added in front of appendix tables

%% Please add \clearpage between each table and/or figure. Further guidelines on figures and tables can be found below.

\authorcontribution{
    Both authors contributed equally.
} %% this section is mandatory

\competinginterests{
    Authors T.O.H. and G.S. are employees of the U.S. Geological Survey.
    } %% this section is mandatory even if you declare that no competing interests are present

\disclaimer{
    Any use of trade, firm, or product names is for descriptive purposes
    only and does not imply endorsement by the U.S. Government.
} %% optional section


\begin{acknowledgements}
    This work was supported by the
    % National Hydrologic Monitoring Program
    % TODO
    Groundwater and Streamflow Information Program
    of the U.S. Geological Survey.
\end{acknowledgements}




%% REFERENCES

%% The reference list is compiled as follows:

%% \begin{thebibliography}{}
%% 
%% \bibitem[AUTHOR(YEAR)]{LABEL1}
%% REFERENCE 1
%% 
%% \bibitem[AUTHOR(YEAR)]{LABEL2}
%% REFERENCE 2
%% 
%% \end{thebibliography}

%% Since the Copernicus LaTeX package includes the BibTeX style file copernicus.bst,
%% authors experienced with BibTeX only have to include the following two lines:
%%
\bibliographystyle{copernicus}
\bibliography{timeseries_error_model.bib}
%%
%% URLs and DOIs can be entered in your BibTeX file as:
%%
%% URL = {http://www.xyz.org/~jones/idx_g.htm}
%% DOI = {10.5194/xyz}


%% LITERATURE CITATIONS
%%
%% command                        & example result
%% \citet{jones90}|               & Jones et al. (1990)
%% \citep{jones90}|               & (Jones et al., 1990)
%% \citep{jones90,jones93}|       & (Jones et al., 1990, 1993)
%% \citep[p.~32]{jones90}|        & (Jones et al., 1990, p.~32)
%% \citep[e.g.,][]{jones90}|      & (e.g., Jones et al., 1990)
%% \citep[e.g.,][p.~32]{jones90}| & (e.g., Jones et al., 1990, p.~32)
%% \citeauthor{jones90}|          & Jones et al.
%% \citeyear{jones90}|            & 1990



%% FIGURES

%% When figures and tables are placed at the end of the MS (article in one-column style), please add \clearpage
%% between bibliography and first table and/or figure as well as between each table and/or figure.

% The figure files should be labelled correctly with Arabic numerals (e.g. fig01.jpg, fig02.png).


%% ONE-COLUMN FIGURES

%%f
%\begin{figure}[t]
%\includegraphics[width=8.3cm]{FILE NAME}
%\caption{TEXT}
%\end{figure}
%
%%% TWO-COLUMN FIGURES
%
%%f
%\begin{figure*}[t]
%\includegraphics[width=12cm]{FILE NAME}
%\caption{TEXT}
%\end{figure*}
%
%
%%% TABLES
%%%
%%% The different columns must be separated with a & command and should
%%% end with \\ to identify the column break.
%
%%% ONE-COLUMN TABLE
%
%%t
%\begin{table}[t]
%\caption{TEXT}
%\begin{tabular}{column = lcr}
%\tophline
%
%\middlehline
%
%\bottomhline
%\end{tabular}
%\belowtable{} % Table Footnotes
%\end{table}
%
%%% TWO-COLUMN TABLE
%
%%t
%\begin{table*}[t]
%\caption{TEXT}
%\begin{tabular}{column = lcr}
%\tophline
%
%\middlehline
%
%\bottomhline
%\end{tabular}
%\belowtable{} % Table Footnotes
%\end{table*}
%
%%% LANDSCAPE TABLE
%
%%t
%\begin{sidewaystable*}[t]
%\caption{TEXT}
%\begin{tabular}{column = lcr}
%\tophline
%
%\middlehline
%
%\bottomhline
%\end{tabular}
%\belowtable{} % Table Footnotes
%\end{sidewaystable*}
%
%
%%% MATHEMATICAL EXPRESSIONS
%
%%% All papers typeset by Copernicus Publications follow the math typesetting regulations
%%% given by the IUPAC Green Book (IUPAC: Quantities, Units and Symbols in Physical Chemistry,
%%% 2nd Edn., Blackwell Science, available at: http://old.iupac.org/publications/books/gbook/green_book_2ed.pdf, 1993).
%%%
%%% Physical quantities/variables are typeset in italic font (t for time, T for Temperature)
%%% Indices which are not defined are typeset in italic font (x, y, z, a, b, c)
%%% Items/objects which are defined are typeset in roman font (Car A, Car B)
%%% Descriptions/specifications which are defined by itself are typeset in roman font (abs, rel, ref, tot, net, ice)
%%% Abbreviations from 2 letters are typeset in roman font (RH, LAI)
%%% Vectors are identified in bold italic font using \vec{x}
%%% Matrices are identified in bold roman font
%%% Multiplication signs are typeset using the LaTeX commands \times (for vector products, grids, and exponential notations) or \cdot
%%% The character * should not be applied as multiplication sign
%
%
%%% EQUATIONS
%
%%% Single-row equation
%
%\begin{equation}
%
%\end{equation}
%
%%% Multiline equation
%
%\begin{align}
%& 3 + 5 = 8\\
%& 3 + 5 = 8\\
%& 3 + 5 = 8
%\end{align}
%
%
%%% MATRICES
%
%\begin{matrix}
%x & y & z\\
%x & y & z\\
%x & y & z\\
%\end{matrix}
%
%
%%% ALGORITHM
%
%\begin{algorithm}
%\caption{...}
%\label{a1}
%\begin{algorithmic}
%...
%\end{algorithmic}
%\end{algorithm}
%
%
%%% CHEMICAL FORMULAS AND REACTIONS
%
%%% For formulas embedded in the text, please use \chem{}
%
%%% The reaction environment creates labels including the letter R, i.e. (R1), (R2), etc.
%
%\begin{reaction}
%%% \rightarrow should be used for normal (one-way) chemical reactions
%%% \rightleftharpoons should be used for equilibria
%%% \leftrightarrow should be used for resonance structures
%\end{reaction}
%
%
%%% PHYSICAL UNITS
%%%
%%% Please use \unit{} and apply the exponential notation (e.g. 20\,\unit{W\,m^{-2}})


\end{document}
