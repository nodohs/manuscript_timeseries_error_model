\documentclass[11pt]{article} % Specifies the document class and an optional font size (10pt is the default)
\usepackage{amsmath} % Example of adding functionality with packages
\usepackage{tabularx}
\usepackage{algorithmic}
\usepackage{algorithm}
\usepackage{amsthm}
\usepackage{float}
\usepackage{subfig}



\title{Uncertainty of Time-Series Measurements with Wiener-Process Fouling and Drift}
\author{
  First A. Author\thanks{USGS, City, Country. Email: first.author@usgs.gov}
  \and
  Second B. Author\thanks{University of Somewhere, City, Country. Email: second.author@uni.edu}
}
% \date{\today} % Sets the date (can be customized or omitted to use the current date)

\begin{document} % Marks the beginning of the document body

\maketitle % Command to display the title, author, and date information

\begin{abstract}
This is the abstract of the article. It provides a brief summary of the content. Test
\end{abstract}

\section{Introduction} % Creates a top-level section heading
This is where you put your content. Paragraphs are separated by one or more blank lines.
You can use \textbf{bold text}, \textit{italics}, and $inline\ math$.

\subsection{Basic Structure} % Creates a sub-section
* Use bullet points for unordered lists.
* Lists are simple to create.

\section{Conclusion}
The document body ends here.

\end{document} % Marks the end of the document body